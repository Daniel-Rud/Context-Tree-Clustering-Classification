% Options for packages loaded elsewhere
\PassOptionsToPackage{unicode}{hyperref}
\PassOptionsToPackage{hyphens}{url}
%
\documentclass[
]{article}
\usepackage{amsmath,amssymb}
\usepackage{iftex}
\ifPDFTeX
  \usepackage[T1]{fontenc}
  \usepackage[utf8]{inputenc}
  \usepackage{textcomp} % provide euro and other symbols
\else % if luatex or xetex
  \usepackage{unicode-math} % this also loads fontspec
  \defaultfontfeatures{Scale=MatchLowercase}
  \defaultfontfeatures[\rmfamily]{Ligatures=TeX,Scale=1}
\fi
\usepackage{lmodern}
\ifPDFTeX\else
  % xetex/luatex font selection
\fi
% Use upquote if available, for straight quotes in verbatim environments
\IfFileExists{upquote.sty}{\usepackage{upquote}}{}
\IfFileExists{microtype.sty}{% use microtype if available
  \usepackage[]{microtype}
  \UseMicrotypeSet[protrusion]{basicmath} % disable protrusion for tt fonts
}{}
\makeatletter
\@ifundefined{KOMAClassName}{% if non-KOMA class
  \IfFileExists{parskip.sty}{%
    \usepackage{parskip}
  }{% else
    \setlength{\parindent}{0pt}
    \setlength{\parskip}{6pt plus 2pt minus 1pt}}
}{% if KOMA class
  \KOMAoptions{parskip=half}}
\makeatother
\usepackage{xcolor}
\usepackage[margin=1in]{geometry}
\usepackage{color}
\usepackage{fancyvrb}
\newcommand{\VerbBar}{|}
\newcommand{\VERB}{\Verb[commandchars=\\\{\}]}
\DefineVerbatimEnvironment{Highlighting}{Verbatim}{commandchars=\\\{\}}
% Add ',fontsize=\small' for more characters per line
\usepackage{framed}
\definecolor{shadecolor}{RGB}{248,248,248}
\newenvironment{Shaded}{\begin{snugshade}}{\end{snugshade}}
\newcommand{\AlertTok}[1]{\textcolor[rgb]{0.94,0.16,0.16}{#1}}
\newcommand{\AnnotationTok}[1]{\textcolor[rgb]{0.56,0.35,0.01}{\textbf{\textit{#1}}}}
\newcommand{\AttributeTok}[1]{\textcolor[rgb]{0.13,0.29,0.53}{#1}}
\newcommand{\BaseNTok}[1]{\textcolor[rgb]{0.00,0.00,0.81}{#1}}
\newcommand{\BuiltInTok}[1]{#1}
\newcommand{\CharTok}[1]{\textcolor[rgb]{0.31,0.60,0.02}{#1}}
\newcommand{\CommentTok}[1]{\textcolor[rgb]{0.56,0.35,0.01}{\textit{#1}}}
\newcommand{\CommentVarTok}[1]{\textcolor[rgb]{0.56,0.35,0.01}{\textbf{\textit{#1}}}}
\newcommand{\ConstantTok}[1]{\textcolor[rgb]{0.56,0.35,0.01}{#1}}
\newcommand{\ControlFlowTok}[1]{\textcolor[rgb]{0.13,0.29,0.53}{\textbf{#1}}}
\newcommand{\DataTypeTok}[1]{\textcolor[rgb]{0.13,0.29,0.53}{#1}}
\newcommand{\DecValTok}[1]{\textcolor[rgb]{0.00,0.00,0.81}{#1}}
\newcommand{\DocumentationTok}[1]{\textcolor[rgb]{0.56,0.35,0.01}{\textbf{\textit{#1}}}}
\newcommand{\ErrorTok}[1]{\textcolor[rgb]{0.64,0.00,0.00}{\textbf{#1}}}
\newcommand{\ExtensionTok}[1]{#1}
\newcommand{\FloatTok}[1]{\textcolor[rgb]{0.00,0.00,0.81}{#1}}
\newcommand{\FunctionTok}[1]{\textcolor[rgb]{0.13,0.29,0.53}{\textbf{#1}}}
\newcommand{\ImportTok}[1]{#1}
\newcommand{\InformationTok}[1]{\textcolor[rgb]{0.56,0.35,0.01}{\textbf{\textit{#1}}}}
\newcommand{\KeywordTok}[1]{\textcolor[rgb]{0.13,0.29,0.53}{\textbf{#1}}}
\newcommand{\NormalTok}[1]{#1}
\newcommand{\OperatorTok}[1]{\textcolor[rgb]{0.81,0.36,0.00}{\textbf{#1}}}
\newcommand{\OtherTok}[1]{\textcolor[rgb]{0.56,0.35,0.01}{#1}}
\newcommand{\PreprocessorTok}[1]{\textcolor[rgb]{0.56,0.35,0.01}{\textit{#1}}}
\newcommand{\RegionMarkerTok}[1]{#1}
\newcommand{\SpecialCharTok}[1]{\textcolor[rgb]{0.81,0.36,0.00}{\textbf{#1}}}
\newcommand{\SpecialStringTok}[1]{\textcolor[rgb]{0.31,0.60,0.02}{#1}}
\newcommand{\StringTok}[1]{\textcolor[rgb]{0.31,0.60,0.02}{#1}}
\newcommand{\VariableTok}[1]{\textcolor[rgb]{0.00,0.00,0.00}{#1}}
\newcommand{\VerbatimStringTok}[1]{\textcolor[rgb]{0.31,0.60,0.02}{#1}}
\newcommand{\WarningTok}[1]{\textcolor[rgb]{0.56,0.35,0.01}{\textbf{\textit{#1}}}}
\usepackage{graphicx}
\makeatletter
\def\maxwidth{\ifdim\Gin@nat@width>\linewidth\linewidth\else\Gin@nat@width\fi}
\def\maxheight{\ifdim\Gin@nat@height>\textheight\textheight\else\Gin@nat@height\fi}
\makeatother
% Scale images if necessary, so that they will not overflow the page
% margins by default, and it is still possible to overwrite the defaults
% using explicit options in \includegraphics[width, height, ...]{}
\setkeys{Gin}{width=\maxwidth,height=\maxheight,keepaspectratio}
% Set default figure placement to htbp
\makeatletter
\def\fps@figure{htbp}
\makeatother
\setlength{\emergencystretch}{3em} % prevent overfull lines
\providecommand{\tightlist}{%
  \setlength{\itemsep}{0pt}\setlength{\parskip}{0pt}}
\setcounter{secnumdepth}{-\maxdimen} % remove section numbering
\ifLuaTeX
  \usepackage{selnolig}  % disable illegal ligatures
\fi
\usepackage{bookmark}
\IfFileExists{xurl.sty}{\usepackage{xurl}}{} % add URL line breaks if available
\urlstyle{same}
\hypersetup{
  pdftitle={Simulation-Code},
  pdfauthor={Daniel Rud},
  hidelinks,
  pdfcreator={LaTeX via pandoc}}

\title{Simulation-Code}
\author{Daniel Rud}
\date{2024-08-21}

\begin{document}
\maketitle

\section{Introduction}\label{introduction}

In this document, we provide code for the simulations described in the
manuscript \emph{Context Tree Clustering and Classification}. We first
introduce the context tree structural dissimilarity measures.

\subsection{Relevant libraries}\label{relevant-libraries}

\begin{Shaded}
\begin{Highlighting}[]
\ControlFlowTok{if}\NormalTok{ (}\SpecialCharTok{!}\FunctionTok{require}\NormalTok{(}\StringTok{"data.tree"}\NormalTok{)) }\FunctionTok{install.packages}\NormalTok{(}\StringTok{"data.tree"}\NormalTok{, }\AttributeTok{dependencies =} \ConstantTok{TRUE}\NormalTok{); }\FunctionTok{library}\NormalTok{(}\StringTok{"data.tree"}\NormalTok{)}
\end{Highlighting}
\end{Shaded}

\begin{verbatim}
## Loading required package: data.tree
\end{verbatim}

\begin{Shaded}
\begin{Highlighting}[]
\ControlFlowTok{if}\NormalTok{ (}\SpecialCharTok{!}\FunctionTok{require}\NormalTok{(}\StringTok{"VLMC"}\NormalTok{)) }\FunctionTok{install.packages}\NormalTok{(}\StringTok{"VLMC"}\NormalTok{, }\AttributeTok{dependencies =} \ConstantTok{TRUE}\NormalTok{); }\FunctionTok{library}\NormalTok{(}\StringTok{"VLMC"}\NormalTok{)}
\end{Highlighting}
\end{Shaded}

\begin{verbatim}
## Loading required package: VLMC
\end{verbatim}

\begin{Shaded}
\begin{Highlighting}[]
\ControlFlowTok{if}\NormalTok{ (}\SpecialCharTok{!}\FunctionTok{require}\NormalTok{(}\StringTok{"stats"}\NormalTok{)) }\FunctionTok{install.packages}\NormalTok{(}\StringTok{"stats"}\NormalTok{, }\AttributeTok{dependencies =} \ConstantTok{TRUE}\NormalTok{); }\FunctionTok{library}\NormalTok{(}\StringTok{"stats"}\NormalTok{)}
\ControlFlowTok{if}\NormalTok{ (}\SpecialCharTok{!}\FunctionTok{require}\NormalTok{(}\StringTok{"gtools"}\NormalTok{)) }\FunctionTok{install.packages}\NormalTok{(}\StringTok{"gtools"}\NormalTok{, }\AttributeTok{dependencies =} \ConstantTok{TRUE}\NormalTok{); }\FunctionTok{library}\NormalTok{(}\StringTok{"gtools"}\NormalTok{)}
\end{Highlighting}
\end{Shaded}

\begin{verbatim}
## Loading required package: gtools
\end{verbatim}

\begin{Shaded}
\begin{Highlighting}[]
\ControlFlowTok{if}\NormalTok{ (}\SpecialCharTok{!}\FunctionTok{require}\NormalTok{(}\StringTok{"cluster"}\NormalTok{)) }\FunctionTok{install.packages}\NormalTok{(}\StringTok{"cluster"}\NormalTok{, }\AttributeTok{dependencies =} \ConstantTok{TRUE}\NormalTok{); }\FunctionTok{library}\NormalTok{(}\StringTok{"cluster"}\NormalTok{)}
\end{Highlighting}
\end{Shaded}

\begin{verbatim}
## Loading required package: cluster
\end{verbatim}

\begin{Shaded}
\begin{Highlighting}[]
\ControlFlowTok{if}\NormalTok{ (}\SpecialCharTok{!}\FunctionTok{require}\NormalTok{(}\StringTok{"factoextra"}\NormalTok{)) }\FunctionTok{install.packages}\NormalTok{(}\StringTok{"factoextra"}\NormalTok{, }\AttributeTok{dependencies =} \ConstantTok{TRUE}\NormalTok{); }\FunctionTok{library}\NormalTok{(}\StringTok{"factoextra"}\NormalTok{)}
\end{Highlighting}
\end{Shaded}

\begin{verbatim}
## Loading required package: factoextra
\end{verbatim}

\begin{verbatim}
## Loading required package: ggplot2
\end{verbatim}

\begin{verbatim}
## Welcome! Want to learn more? See two factoextra-related books at https://goo.gl/ve3WBa
\end{verbatim}

\begin{Shaded}
\begin{Highlighting}[]
\ControlFlowTok{if}\NormalTok{ (}\SpecialCharTok{!}\FunctionTok{require}\NormalTok{(}\StringTok{"mclust"}\NormalTok{)) }\FunctionTok{install.packages}\NormalTok{(}\StringTok{"mclust"}\NormalTok{, }\AttributeTok{dependencies =} \ConstantTok{TRUE}\NormalTok{); }\FunctionTok{library}\NormalTok{(}\StringTok{"mclust"}\NormalTok{)}
\end{Highlighting}
\end{Shaded}

\begin{verbatim}
## Loading required package: mclust
\end{verbatim}

\begin{verbatim}
## Package 'mclust' version 6.1.1
## Type 'citation("mclust")' for citing this R package in publications.
\end{verbatim}

\begin{Shaded}
\begin{Highlighting}[]
\ControlFlowTok{if}\NormalTok{ (}\SpecialCharTok{!}\FunctionTok{require}\NormalTok{(}\StringTok{"infotheo"}\NormalTok{)) }\FunctionTok{install.packages}\NormalTok{(}\StringTok{"infotheo"}\NormalTok{, }\AttributeTok{dependencies =} \ConstantTok{TRUE}\NormalTok{); }\FunctionTok{library}\NormalTok{(}\StringTok{"infotheo"}\NormalTok{)}
\end{Highlighting}
\end{Shaded}

\begin{verbatim}
## Loading required package: infotheo
\end{verbatim}

\begin{verbatim}
## 
## Attaching package: 'infotheo'
\end{verbatim}

\begin{verbatim}
## The following object is masked from 'package:VLMC':
## 
##     entropy
\end{verbatim}

\begin{Shaded}
\begin{Highlighting}[]
\ControlFlowTok{if}\NormalTok{ (}\SpecialCharTok{!}\FunctionTok{require}\NormalTok{(}\StringTok{"fossil"}\NormalTok{)) }\FunctionTok{install.packages}\NormalTok{(}\StringTok{"fossil"}\NormalTok{, }\AttributeTok{dependencies =} \ConstantTok{TRUE}\NormalTok{); }\FunctionTok{library}\NormalTok{(}\StringTok{"fossil"}\NormalTok{)}
\end{Highlighting}
\end{Shaded}

\begin{verbatim}
## Loading required package: fossil
\end{verbatim}

\begin{verbatim}
## Loading required package: sp
\end{verbatim}

\begin{verbatim}
## Loading required package: maps
\end{verbatim}

\begin{verbatim}
## 
## Attaching package: 'maps'
\end{verbatim}

\begin{verbatim}
## The following object is masked from 'package:mclust':
## 
##     map
\end{verbatim}

\begin{verbatim}
## The following object is masked from 'package:cluster':
## 
##     votes.repub
\end{verbatim}

\begin{verbatim}
## Loading required package: shapefiles
\end{verbatim}

\begin{verbatim}
## Loading required package: foreign
\end{verbatim}

\begin{verbatim}
## 
## Attaching package: 'shapefiles'
\end{verbatim}

\begin{verbatim}
## The following objects are masked from 'package:foreign':
## 
##     read.dbf, write.dbf
\end{verbatim}

\section{Loading Neccessary
Functions}\label{loading-neccessary-functions}

We first need some relevant helper functions.

\begin{Shaded}
\begin{Highlighting}[]
\CommentTok{\# reverse a string}
\NormalTok{strReverse }\OtherTok{\textless{}{-}} \ControlFlowTok{function}\NormalTok{(x)}
  \FunctionTok{sapply}\NormalTok{(}\FunctionTok{lapply}\NormalTok{(}\FunctionTok{strsplit}\NormalTok{(x, }\ConstantTok{NULL}\NormalTok{), rev), paste, }\AttributeTok{collapse=}\StringTok{""}\NormalTok{)}


\CommentTok{\# generate all state space paths from a VLMC tree object }
\NormalTok{vlmcTreePaths}\OtherTok{=} \ControlFlowTok{function}\NormalTok{(vlmcObj)}
\NormalTok{\{}
\NormalTok{  dendrogram}\OtherTok{=} \FunctionTok{as.dendrogram}\NormalTok{(vlmcObj)}
  \CommentTok{\#plot(dendrogram)}
\NormalTok{  allPaths}\OtherTok{=}\FunctionTok{vector}\NormalTok{(}\AttributeTok{mode=}\StringTok{"character"}\NormalTok{)}
  \ControlFlowTok{for}\NormalTok{(i }\ControlFlowTok{in} \DecValTok{1}\SpecialCharTok{:}\FunctionTok{length}\NormalTok{(}\FunctionTok{names}\NormalTok{(dendrogram)))}
\NormalTok{  \{}
\NormalTok{    string}\OtherTok{=} \FunctionTok{capture.output}\NormalTok{(dendrogram[i])}
\NormalTok{    entries}\OtherTok{=} \FunctionTok{which}\NormalTok{(}\FunctionTok{grepl}\NormalTok{(}\StringTok{"$"}\NormalTok{, string, }\AttributeTok{fixed=}\ConstantTok{TRUE}\NormalTok{))}
\NormalTok{    paths}\OtherTok{=}\NormalTok{ string[entries]}
    \ControlFlowTok{for}\NormalTok{( j }\ControlFlowTok{in} \DecValTok{1}\SpecialCharTok{:} \FunctionTok{length}\NormalTok{(paths))}
\NormalTok{    \{}
\NormalTok{      paths[j]}\OtherTok{=} \FunctionTok{gsub}\NormalTok{(}\StringTok{"$"}\NormalTok{, }\StringTok{""}\NormalTok{, paths[j], }\AttributeTok{fixed=}\ConstantTok{TRUE}\NormalTok{)}
\NormalTok{      paths[j]}\OtherTok{=} \FunctionTok{gsub}\NormalTok{(}\StringTok{"\textasciigrave{}"}\NormalTok{, }\StringTok{""}\NormalTok{, paths[j], }\AttributeTok{fixed=}\ConstantTok{TRUE}\NormalTok{)}
\NormalTok{      paths[j]}\OtherTok{=} \FunctionTok{gsub}\NormalTok{(}\StringTok{"NA"}\NormalTok{, }\StringTok{""}\NormalTok{, paths[j], }\AttributeTok{fixed=}\ConstantTok{TRUE}\NormalTok{)}
\NormalTok{    \}}
\NormalTok{    allPaths}\OtherTok{=}\FunctionTok{append}\NormalTok{(allPaths, paths)}
\NormalTok{  \}}
  \FunctionTok{return}\NormalTok{(allPaths)  }
\NormalTok{\}}
\end{Highlighting}
\end{Shaded}

Now, we define the structural distance tDistance, the probability based
dissimilarity measure pDistance, and the

\begin{Shaded}
\begin{Highlighting}[]
\CommentTok{\# structural distance measure }
\NormalTok{tDistance}\OtherTok{=} \ControlFlowTok{function}\NormalTok{(VLMC1, VLMC2)}
\NormalTok{\{}
\NormalTok{  Pa}\OtherTok{=} \FunctionTok{vlmcTreePaths}\NormalTok{(VLMC1)}
  \CommentTok{\#x11()}
\NormalTok{  Pb}\OtherTok{=} \FunctionTok{vlmcTreePaths}\NormalTok{(VLMC2)}
  
\NormalTok{  diffAB}\OtherTok{=} \FunctionTok{setdiff}\NormalTok{(Pa, Pb) }\CommentTok{\#elements in Pa but not Pb}
\NormalTok{  diffBA}\OtherTok{=} \FunctionTok{setdiff}\NormalTok{(Pb, Pa) }\CommentTok{\#elements in Pb but not Pa}
\NormalTok{  distance}\OtherTok{=}\DecValTok{0}
  \ControlFlowTok{if}\NormalTok{ (}\FunctionTok{length}\NormalTok{(diffAB) }\SpecialCharTok{\textgreater{}} \DecValTok{0}\NormalTok{)}
\NormalTok{  \{}
    \ControlFlowTok{for}\NormalTok{(i }\ControlFlowTok{in} \DecValTok{1}\SpecialCharTok{:}\FunctionTok{length}\NormalTok{(diffAB))}
\NormalTok{    \{}
\NormalTok{      distance}\OtherTok{=}\NormalTok{distance}\SpecialCharTok{+} \DecValTok{1}\SpecialCharTok{/}\FunctionTok{nchar}\NormalTok{(diffAB[i])}
\NormalTok{    \}}
\NormalTok{  \}}
  \ControlFlowTok{if}\NormalTok{ (}\FunctionTok{length}\NormalTok{(diffBA) }\SpecialCharTok{\textgreater{}} \DecValTok{0}\NormalTok{)}
\NormalTok{  \{}
    \ControlFlowTok{for}\NormalTok{(i }\ControlFlowTok{in} \DecValTok{1}\SpecialCharTok{:}\FunctionTok{length}\NormalTok{(diffBA))}
\NormalTok{    \{}
\NormalTok{      distance}\OtherTok{=}\NormalTok{ distance}\SpecialCharTok{+} \DecValTok{1}\SpecialCharTok{/}\FunctionTok{nchar}\NormalTok{(diffBA[i])}
\NormalTok{    \}}
\NormalTok{  \}}
  
  \FunctionTok{return}\NormalTok{(distance)}
\NormalTok{\}}

\CommentTok{\# distance based on the structure and the probabilities {-} }
\CommentTok{\# need the data (textOutput1 and 2) to estimate the probabilities }
\NormalTok{pDistance }\OtherTok{\textless{}{-}} \ControlFlowTok{function}\NormalTok{(VLMC1, VLMC2, textOutput1, textOutput2)}
\NormalTok{\{}
\NormalTok{  Pa }\OtherTok{=} \FunctionTok{vlmcTreePaths}\NormalTok{(VLMC1)}
\NormalTok{  Pb }\OtherTok{=} \FunctionTok{vlmcTreePaths}\NormalTok{(VLMC2) }
\NormalTok{  union }\OtherTok{=} \FunctionTok{intersect}\NormalTok{(Pa, Pb)}
  
  \CommentTok{\#find distance between transition probabilities}
\NormalTok{  states }\OtherTok{=} \FunctionTok{as.character}\NormalTok{(}\FunctionTok{sort}\NormalTok{(}\FunctionTok{unique}\NormalTok{(}\FunctionTok{c}\NormalTok{(textOutput1, textOutput2))))}
\NormalTok{  pDist }\OtherTok{=} \DecValTok{0}
\NormalTok{  text1 }\OtherTok{=} \FunctionTok{paste}\NormalTok{(textOutput1, }\AttributeTok{collapse=}\StringTok{""}\NormalTok{)}
\NormalTok{  text2 }\OtherTok{=} \FunctionTok{paste}\NormalTok{(textOutput2, }\AttributeTok{collapse=}\StringTok{""}\NormalTok{)}
  
  \ControlFlowTok{if}\NormalTok{ (}\FunctionTok{length}\NormalTok{(union) }\SpecialCharTok{\textgreater{}} \DecValTok{0}\NormalTok{) }\DocumentationTok{\#\# one of the trees is just the root {-}\textgreater{} order 0}
    \ControlFlowTok{for}\NormalTok{(i }\ControlFlowTok{in} \DecValTok{1}\SpecialCharTok{:} \FunctionTok{length}\NormalTok{(union))}
\NormalTok{    \{}
      
\NormalTok{      sequence }\OtherTok{=} \FunctionTok{strReverse}\NormalTok{(union[i])}
\NormalTok{      occurences1 }\OtherTok{=} \FunctionTok{gregexpr}\NormalTok{(sequence,}\FunctionTok{substr}\NormalTok{(text1,}\DecValTok{1}\NormalTok{,(}\FunctionTok{nchar}\NormalTok{(text1)}\SpecialCharTok{{-}}\DecValTok{1}\NormalTok{)))[[}\DecValTok{1}\NormalTok{]] }\CommentTok{\#location where the path occurs  {-} the substr avoids the last observation (edge case there is no "next")}
\NormalTok{      occurences2 }\OtherTok{=} \FunctionTok{gregexpr}\NormalTok{(sequence,}\FunctionTok{substr}\NormalTok{(text2,}\DecValTok{1}\NormalTok{,(}\FunctionTok{nchar}\NormalTok{(text2)}\SpecialCharTok{{-}}\DecValTok{1}\NormalTok{)))[[}\DecValTok{1}\NormalTok{]]}
\NormalTok{      next1 }\OtherTok{=}\NormalTok{ occurences1}\SpecialCharTok{+}\FunctionTok{nchar}\NormalTok{(sequence) }\CommentTok{\#index of next term (for probabilities)}
\NormalTok{      next2 }\OtherTok{=}\NormalTok{ occurences2}\SpecialCharTok{+}\FunctionTok{nchar}\NormalTok{(sequence)}
      
      
      
\NormalTok{      nextVal1 }\OtherTok{=} \FunctionTok{as.vector}\NormalTok{(}\FunctionTok{strsplit}\NormalTok{(text1, }\AttributeTok{split=}\ConstantTok{NULL}\NormalTok{)[[}\DecValTok{1}\NormalTok{]])[next1] }\CommentTok{\#finds actual next value after pattern }
\NormalTok{      nextVal2 }\OtherTok{=} \FunctionTok{as.vector}\NormalTok{(}\FunctionTok{strsplit}\NormalTok{(text2, }\AttributeTok{split=}\ConstantTok{NULL}\NormalTok{)[[}\DecValTok{1}\NormalTok{]])[next2]}
\NormalTok{      counts1 }\OtherTok{=} \FunctionTok{table}\NormalTok{(nextVal1)}
\NormalTok{      counts2 }\OtherTok{=} \FunctionTok{table}\NormalTok{(nextVal2)     }
      
      \ControlFlowTok{if}\NormalTok{ (}\SpecialCharTok{!}\FunctionTok{identical}\NormalTok{(}\FunctionTok{names}\NormalTok{(}\FunctionTok{table}\NormalTok{(nextVal1)), }\FunctionTok{names}\NormalTok{(}\FunctionTok{table}\NormalTok{(nextVal2)))) }\DocumentationTok{\#\# not all alphabet is in next symbols}
\NormalTok{      \{ }
\NormalTok{        counts1\_aux }\OtherTok{=} \FunctionTok{rep}\NormalTok{(}\DecValTok{0}\NormalTok{,}\DecValTok{5}\NormalTok{)}
\NormalTok{        counts2\_aux }\OtherTok{=} \FunctionTok{rep}\NormalTok{(}\DecValTok{0}\NormalTok{,}\DecValTok{5}\NormalTok{)}
\NormalTok{        elements }\OtherTok{=} \FunctionTok{c}\NormalTok{(}\StringTok{"0"}\NormalTok{,}\StringTok{"1"}\NormalTok{,}\StringTok{"2"}\NormalTok{,}\StringTok{"3"}\NormalTok{,}\StringTok{"4"}\NormalTok{)}
        \ControlFlowTok{for}\NormalTok{ (j }\ControlFlowTok{in} \DecValTok{1}\SpecialCharTok{:}\FunctionTok{length}\NormalTok{(elements))}
\NormalTok{        \{}
\NormalTok{          counts1\_aux[j]  }\OtherTok{=}\NormalTok{ counts1[elements[j]]}
\NormalTok{          counts2\_aux[j]  }\OtherTok{=}\NormalTok{ counts2[elements[j]]                }
\NormalTok{        \}}
        
\NormalTok{        counts1\_aux[}\FunctionTok{is.na}\NormalTok{(counts1\_aux)] }\OtherTok{=} \DecValTok{0}
\NormalTok{        counts1 }\OtherTok{=}\NormalTok{ counts1\_aux}
\NormalTok{        counts2\_aux[}\FunctionTok{is.na}\NormalTok{(counts2\_aux)] }\OtherTok{=} \DecValTok{0}
\NormalTok{        counts2 }\OtherTok{=}\NormalTok{ counts2\_aux}
\NormalTok{      \}}
      
\NormalTok{      pDist}\OtherTok{=}\NormalTok{pDist }\SpecialCharTok{+} \FunctionTok{sum}\NormalTok{(}\FunctionTok{abs}\NormalTok{((counts1}\SpecialCharTok{/}\FunctionTok{sum}\NormalTok{(counts1))}\SpecialCharTok{{-}}\NormalTok{(counts2}\SpecialCharTok{/}\FunctionTok{sum}\NormalTok{(counts2))))}
\NormalTok{    \}}
  
  \FunctionTok{return}\NormalTok{(pDist)}
\NormalTok{\}}
\end{Highlighting}
\end{Shaded}

Functions for KNN

\begin{Shaded}
\begin{Highlighting}[]
\DocumentationTok{\#\# classical KNN }
\NormalTok{knn }\OtherTok{\textless{}{-}} \ControlFlowTok{function}\NormalTok{(data\_set\_train, data\_set\_test, true\_labels\_train, n.neighbors)}
\NormalTok{\{}
\NormalTok{  result }\OtherTok{=} \DecValTok{0}
  \ControlFlowTok{for}\NormalTok{ (i }\ControlFlowTok{in} \DecValTok{1}\SpecialCharTok{:}\FunctionTok{length}\NormalTok{(data\_set\_test[,}\DecValTok{1}\NormalTok{]))}
\NormalTok{  \{}
\NormalTok{    dist\_to\_train\_data }\OtherTok{=} \FunctionTok{rowSums}\NormalTok{(}\FunctionTok{abs}\NormalTok{(}\FunctionTok{sweep}\NormalTok{(data\_set\_train, }\DecValTok{2}\NormalTok{, data\_set\_test[i,])))}
\NormalTok{    which\_train }\OtherTok{=} \FunctionTok{order}\NormalTok{(dist\_to\_train\_data)[}\DecValTok{1}\SpecialCharTok{:}\NormalTok{n.neighbors]}
\NormalTok{    majority }\OtherTok{=} \DecValTok{0}
    \ControlFlowTok{for}\NormalTok{ (j }\ControlFlowTok{in}\NormalTok{ which\_train)}
\NormalTok{      majority }\OtherTok{=}\NormalTok{ majority }\SpecialCharTok{+} \FunctionTok{ifelse}\NormalTok{(true\_labels\_train[j] }\SpecialCharTok{==} \DecValTok{1}\NormalTok{,}\DecValTok{1}\NormalTok{,}\SpecialCharTok{{-}}\DecValTok{1}\NormalTok{)}
    
    \ControlFlowTok{if}\NormalTok{ (majority }\SpecialCharTok{==} \DecValTok{0}\NormalTok{)}
\NormalTok{      result[i] }\OtherTok{=} \FunctionTok{rbinom}\NormalTok{(}\DecValTok{1}\NormalTok{,}\DecValTok{1}\NormalTok{,.}\DecValTok{5}\NormalTok{)}
    \ControlFlowTok{else}
\NormalTok{      result[i] }\OtherTok{=} \FunctionTok{ifelse}\NormalTok{(majority }\SpecialCharTok{\textgreater{}} \DecValTok{0}\NormalTok{, }\DecValTok{1}\NormalTok{, }\DecValTok{0}\NormalTok{)}
\NormalTok{  \}}
  \FunctionTok{return}\NormalTok{(result)}
\NormalTok{\}}



\NormalTok{KNN\_dist\_matrix }\OtherTok{\textless{}{-}} \ControlFlowTok{function}\NormalTok{(fulldistanceMatrix, train\_sample\_index, test\_sample\_index, all\_labels, n.neighbors)}
\NormalTok{\{}
  \CommentTok{\# Performs KNN classification given distance matrix}
  
\NormalTok{  result }\OtherTok{=} \DecValTok{0}
  \ControlFlowTok{for}\NormalTok{ (i }\ControlFlowTok{in} \DecValTok{1}\SpecialCharTok{:}\FunctionTok{length}\NormalTok{(test\_sample\_index))}
\NormalTok{  \{}
\NormalTok{    distances }\OtherTok{=}\NormalTok{ fulldistanceMatrix[,test\_sample\_index[i]]}
\NormalTok{    dist\_to\_train\_data }\OtherTok{=}\NormalTok{ distances[train\_sample\_index]}
\NormalTok{    order\_dist }\OtherTok{=} \FunctionTok{order}\NormalTok{(dist\_to\_train\_data)[}\DecValTok{1}\SpecialCharTok{:}\NormalTok{n.neighbors]}
\NormalTok{    which\_train }\OtherTok{=}\NormalTok{ train\_sample\_index[order\_dist]}
    
\NormalTok{    result[i] }\OtherTok{=} \FunctionTok{as.numeric}\NormalTok{(}\FunctionTok{majorityVote}\NormalTok{(all\_labels[which\_train])}\SpecialCharTok{$}\NormalTok{majority)}
\NormalTok{  \}}
  
  \DocumentationTok{\#\# returns a vector of the classification for each test data}
  \FunctionTok{return}\NormalTok{(result)}
\NormalTok{\}}
\end{Highlighting}
\end{Shaded}

We now define a function to measure cluster accuracy

\begin{Shaded}
\begin{Highlighting}[]
\NormalTok{Mismatch }\OtherTok{\textless{}{-}} \ControlFlowTok{function}\NormalTok{(mm, true\_clusters, K)}
\NormalTok{\{}
\NormalTok{  sigma }\OtherTok{=} \FunctionTok{permutations}\NormalTok{(}\AttributeTok{n=}\NormalTok{K,}\AttributeTok{r=}\NormalTok{K,}\AttributeTok{v=}\DecValTok{1}\SpecialCharTok{:}\NormalTok{K)}
  
\NormalTok{  Miss }\OtherTok{=} \FunctionTok{length}\NormalTok{(}\FunctionTok{which}\NormalTok{( true\_clusters }\SpecialCharTok{!=}\NormalTok{ mm))  }\DocumentationTok{\#\# for permutation 1, 2,... K}
  
\NormalTok{  mm\_aux }\OtherTok{=}\NormalTok{ mm}
  \ControlFlowTok{for}\NormalTok{ (ind }\ControlFlowTok{in} \DecValTok{2}\SpecialCharTok{:}\FunctionTok{dim}\NormalTok{(sigma)[}\DecValTok{1}\NormalTok{])}
\NormalTok{  \{}
    
    \ControlFlowTok{for}\NormalTok{ (j }\ControlFlowTok{in} \DecValTok{1}\SpecialCharTok{:}\NormalTok{K)}
\NormalTok{      mm\_aux[}\FunctionTok{which}\NormalTok{(mm }\SpecialCharTok{==}\NormalTok{ j)] }\OtherTok{=}\NormalTok{ sigma[ind,j]}
    
\NormalTok{    Miss[ind] }\OtherTok{=}  \FunctionTok{length}\NormalTok{(}\FunctionTok{which}\NormalTok{( true\_clusters }\SpecialCharTok{!=}\NormalTok{ mm\_aux))}
    
\NormalTok{  \}}
  
  
  \FunctionTok{return}\NormalTok{(}\FunctionTok{min}\NormalTok{(Miss))}
\NormalTok{\}}
\end{Highlighting}
\end{Shaded}

\section{Simulation Study:}\label{simulation-study}

First, we define two generating VLMCs from two seperate state sequences.
vlmcA and vlmcB serve as the two \emph{population} context trees from
which we will simulate state sequences from.

\begin{Shaded}
\begin{Highlighting}[]
\NormalTok{dataA }\OtherTok{=} \FunctionTok{c}\NormalTok{(}\DecValTok{1}\NormalTok{, }\DecValTok{3}\NormalTok{, }\DecValTok{3}\NormalTok{, }\DecValTok{3}\NormalTok{, }\DecValTok{0}\NormalTok{, }\DecValTok{0}\NormalTok{, }\DecValTok{2}\NormalTok{, }\DecValTok{1}\NormalTok{, }\DecValTok{3}\NormalTok{, }\DecValTok{3}\NormalTok{, }\DecValTok{3}\NormalTok{, }\DecValTok{2}\NormalTok{, }\DecValTok{1}\NormalTok{, }\DecValTok{0}\NormalTok{, }\DecValTok{0}\NormalTok{, }\DecValTok{3}\NormalTok{, }\DecValTok{0}\NormalTok{, }\DecValTok{4}\NormalTok{, }\DecValTok{3}\NormalTok{, }\DecValTok{2}\NormalTok{, }\DecValTok{1}\NormalTok{, }\DecValTok{2}\NormalTok{, }\DecValTok{1}\NormalTok{, }\DecValTok{0}\NormalTok{, }\DecValTok{0}\NormalTok{, }\DecValTok{4}\NormalTok{, }\DecValTok{2}\NormalTok{, }\DecValTok{1}\NormalTok{, }\DecValTok{0}\NormalTok{, }\DecValTok{3}\NormalTok{, }\DecValTok{3}\NormalTok{, }\DecValTok{0}\NormalTok{, }\DecValTok{3}\NormalTok{, }\DecValTok{3}\NormalTok{, }\DecValTok{3}\NormalTok{, }\DecValTok{3}\NormalTok{, }\DecValTok{0}\NormalTok{, }\DecValTok{0}\NormalTok{, }\DecValTok{2}\NormalTok{, }\DecValTok{1}\NormalTok{, }\DecValTok{3}\NormalTok{, }\DecValTok{0}\NormalTok{, }\DecValTok{0}\NormalTok{, }\DecValTok{3}\NormalTok{, }\DecValTok{3}\NormalTok{, }\DecValTok{3}\NormalTok{, }\DecValTok{2}\NormalTok{, }\DecValTok{1}\NormalTok{, }\DecValTok{0}\NormalTok{, }\DecValTok{0}\NormalTok{, }\DecValTok{3}\NormalTok{, }\DecValTok{0}\NormalTok{, }\DecValTok{3}\NormalTok{, }\DecValTok{3}\NormalTok{, }\DecValTok{0}\NormalTok{, }\DecValTok{3}\NormalTok{, }\DecValTok{0}\NormalTok{, }\DecValTok{3}\NormalTok{, }\DecValTok{3}\NormalTok{, }\DecValTok{0}\NormalTok{, }\DecValTok{3}\NormalTok{, }\DecValTok{0}\NormalTok{, }\DecValTok{3}\NormalTok{, }\DecValTok{0}\NormalTok{, }\DecValTok{3}\NormalTok{, }\DecValTok{4}\NormalTok{, }\DecValTok{3}\NormalTok{, }\DecValTok{2}\NormalTok{, }\DecValTok{1}\NormalTok{, }\DecValTok{0}\NormalTok{, }\DecValTok{0}\NormalTok{, }\DecValTok{3}\NormalTok{, }\DecValTok{2}\NormalTok{, }\DecValTok{1}\NormalTok{, }\DecValTok{2}\NormalTok{, }\DecValTok{1}\NormalTok{, }\DecValTok{0}\NormalTok{, }\DecValTok{3}\NormalTok{, }\DecValTok{3}\NormalTok{, }\DecValTok{3}\NormalTok{, }\DecValTok{3}\NormalTok{, }\DecValTok{2}\NormalTok{, }\DecValTok{1}\NormalTok{, }\DecValTok{3}\NormalTok{, }\DecValTok{3}\NormalTok{, }\DecValTok{0}\NormalTok{, }\DecValTok{2}\NormalTok{, }\DecValTok{1}\NormalTok{, }\DecValTok{4}\NormalTok{, }\DecValTok{3}\NormalTok{, }\DecValTok{0}\NormalTok{, }\DecValTok{3}\NormalTok{, }\DecValTok{3}\NormalTok{, }\DecValTok{3}\NormalTok{, }\DecValTok{3}\NormalTok{, }\DecValTok{4}\NormalTok{, }\DecValTok{3}\NormalTok{, }\DecValTok{0}\NormalTok{, }\DecValTok{3}\NormalTok{, }\DecValTok{3}\NormalTok{)}
\FunctionTok{draw}\NormalTok{(}\FunctionTok{vlmc}\NormalTok{(dataA))}
\end{Highlighting}
\end{Shaded}

\begin{verbatim}
## [x]-( 27 12 12 43 5| 99)
##  +--[2]-( 0 12 0 0 0| 12) <25.32>-T
##  '--[3]-( 14 0 6 20 2| 42) <5.52>-T
\end{verbatim}

\begin{Shaded}
\begin{Highlighting}[]
\NormalTok{vlmcA }\OtherTok{=} \FunctionTok{vlmc}\NormalTok{(dataA)}


\NormalTok{dataB }\OtherTok{=} \FunctionTok{c}\NormalTok{(}\DecValTok{3}\NormalTok{, }\DecValTok{1}\NormalTok{, }\DecValTok{3}\NormalTok{, }\DecValTok{1}\NormalTok{, }\DecValTok{3}\NormalTok{, }\DecValTok{3}\NormalTok{, }\DecValTok{3}\NormalTok{, }\DecValTok{0}\NormalTok{, }\DecValTok{2}\NormalTok{, }\DecValTok{1}\NormalTok{, }\DecValTok{2}\NormalTok{, }\DecValTok{1}\NormalTok{, }\DecValTok{3}\NormalTok{, }\DecValTok{3}\NormalTok{, }\DecValTok{3}\NormalTok{, }\DecValTok{3}\NormalTok{, }\DecValTok{3}\NormalTok{, }\DecValTok{4}\NormalTok{, }\DecValTok{3}\NormalTok{, }\DecValTok{0}\NormalTok{, }\DecValTok{3}\NormalTok{, }\DecValTok{3}\NormalTok{, }\DecValTok{0}\NormalTok{, }\DecValTok{3}\NormalTok{, }\DecValTok{3}\NormalTok{, }\DecValTok{3}\NormalTok{, }\DecValTok{0}\NormalTok{, }\DecValTok{0}\NormalTok{, }\DecValTok{3}\NormalTok{, }\DecValTok{0}\NormalTok{, }\DecValTok{3}\NormalTok{, }\DecValTok{0}\NormalTok{, }\DecValTok{3}\NormalTok{, }\DecValTok{2}\NormalTok{, }\DecValTok{1}\NormalTok{, }\DecValTok{0}\NormalTok{, }\DecValTok{0}\NormalTok{, }\DecValTok{3}\NormalTok{, }\DecValTok{3}\NormalTok{, }\DecValTok{3}\NormalTok{, }\DecValTok{0}\NormalTok{, }\DecValTok{2}\NormalTok{, }\DecValTok{0}\NormalTok{, }\DecValTok{1}\NormalTok{, }\DecValTok{0}\NormalTok{, }\DecValTok{2}\NormalTok{, }\DecValTok{1}\NormalTok{, }\DecValTok{0}\NormalTok{, }\DecValTok{0}\NormalTok{, }\DecValTok{3}\NormalTok{, }\DecValTok{2}\NormalTok{, }\DecValTok{0}\NormalTok{, }\DecValTok{1}\NormalTok{, }\DecValTok{2}\NormalTok{, }\DecValTok{0}\NormalTok{, }\DecValTok{1}\NormalTok{, }\DecValTok{0}\NormalTok{, }\DecValTok{4}\NormalTok{, }\DecValTok{3}\NormalTok{, }\DecValTok{0}\NormalTok{, }\DecValTok{3}\NormalTok{, }\DecValTok{0}\NormalTok{, }\DecValTok{2}\NormalTok{, }\DecValTok{1}\NormalTok{, }\DecValTok{4}\NormalTok{, }\DecValTok{3}\NormalTok{, }\DecValTok{1}\NormalTok{, }\DecValTok{1}\NormalTok{, }\DecValTok{0}\NormalTok{, }\DecValTok{3}\NormalTok{, }\DecValTok{3}\NormalTok{, }\DecValTok{3}\NormalTok{, }\DecValTok{3}\NormalTok{, }\DecValTok{2}\NormalTok{, }\DecValTok{1}\NormalTok{, }\DecValTok{3}\NormalTok{, }\DecValTok{3}\NormalTok{, }\DecValTok{3}\NormalTok{, }\DecValTok{3}\NormalTok{, }\DecValTok{3}\NormalTok{, }\DecValTok{0}\NormalTok{, }\DecValTok{3}\NormalTok{, }\DecValTok{3}\NormalTok{, }\DecValTok{2}\NormalTok{, }\DecValTok{1}\NormalTok{, }\DecValTok{3}\NormalTok{, }\DecValTok{2}\NormalTok{, }\DecValTok{1}\NormalTok{, }\DecValTok{3}\NormalTok{, }\DecValTok{0}\NormalTok{, }\DecValTok{3}\NormalTok{, }\DecValTok{0}\NormalTok{, }\DecValTok{0}\NormalTok{, }\DecValTok{3}\NormalTok{, }\DecValTok{4}\NormalTok{, }\DecValTok{3}\NormalTok{, }\DecValTok{2}\NormalTok{, }\DecValTok{1}\NormalTok{, }\DecValTok{4}\NormalTok{, }\DecValTok{3}\NormalTok{, }\DecValTok{0}\NormalTok{, }\DecValTok{3}\NormalTok{, }\DecValTok{3}\NormalTok{, }\DecValTok{0}\NormalTok{, }\DecValTok{4}\NormalTok{, }\DecValTok{3}\NormalTok{, }\DecValTok{0}\NormalTok{, }\DecValTok{0}\NormalTok{, }\DecValTok{3}\NormalTok{, }\DecValTok{0}\NormalTok{, }\DecValTok{3}\NormalTok{, }\DecValTok{3}\NormalTok{, }\DecValTok{2}\NormalTok{, }\DecValTok{1}\NormalTok{, }\DecValTok{0}\NormalTok{, }\DecValTok{3}\NormalTok{, }\DecValTok{2}\NormalTok{, }\DecValTok{1}\NormalTok{, }\DecValTok{3}\NormalTok{, }\DecValTok{3}\NormalTok{, }\DecValTok{3}\NormalTok{, }\DecValTok{2}\NormalTok{, }\DecValTok{1}\NormalTok{, }\DecValTok{0}\NormalTok{, }\DecValTok{3}\NormalTok{, }\DecValTok{3}\NormalTok{, }\DecValTok{4}\NormalTok{, }\DecValTok{2}\NormalTok{, }\DecValTok{1}\NormalTok{, }\DecValTok{3}\NormalTok{, }\DecValTok{4}\NormalTok{, }\DecValTok{3}\NormalTok{, }\DecValTok{2}\NormalTok{, }\DecValTok{1}\NormalTok{, }\DecValTok{3}\NormalTok{, }\DecValTok{2}\NormalTok{, }\DecValTok{0}\NormalTok{, }\DecValTok{1}\NormalTok{, }\DecValTok{3}\NormalTok{, }\DecValTok{2}\NormalTok{)}
\FunctionTok{draw}\NormalTok{(}\FunctionTok{vlmc}\NormalTok{(dataB))}
\end{Highlighting}
\end{Shaded}

\begin{verbatim}
## [x]-( 32 22 19 58 8| 139)
##  +--[0]-( 5 4 4 17 2| 32) <1.03>
##  |   '--[2]-( 0 4 0 0 0| 4) <8.32>-T
##  +--[2]-( 4 14 0 0 0| 18) <22.15>-T
##  '--[4]-( 0 0 1 7 0| 8) <5.09>-T
\end{verbatim}

\begin{Shaded}
\begin{Highlighting}[]
\NormalTok{vlmcB }\OtherTok{=} \FunctionTok{vlmc}\NormalTok{(dataB)}
\end{Highlighting}
\end{Shaded}

The simulations in Table 1. iterate over the length of the state
sequence sampled from each of the 40 vlmcA and 40 vlmcB context trees.
We showcase a single iteration, where the length of the state sequence
is 500.

\begin{Shaded}
\begin{Highlighting}[]
\NormalTok{n.time\_series }\OtherTok{=} \DecValTok{500} \DocumentationTok{\#\# how many observations of the chain we are generating}
\NormalTok{all\_results }\OtherTok{=} \FunctionTok{list}\NormalTok{() }\DocumentationTok{\#\# each element will hold all results of the simulation for a specific time series n}

\NormalTok{n }\OtherTok{=} \DecValTok{80} \CommentTok{\# total number of VLMC observations {-}{-} 40 + 40 }
\NormalTok{n.neighbors }\OtherTok{=} \DecValTok{7} \CommentTok{\# number of neighbors for classification }
\NormalTok{vlmcs }\OtherTok{=} \FunctionTok{list}\NormalTok{()}

\CommentTok{\# for reproducability}
\FunctionTok{set.seed}\NormalTok{(}\DecValTok{1}\NormalTok{)}

\CommentTok{\# number of unique classes K}
\NormalTok{K }\OtherTok{=} \DecValTok{2}   
\end{Highlighting}
\end{Shaded}

Now, we begin the simulations. First, we simulate the VLMC state
sequences from vlmcA and vlmcB.

\begin{Shaded}
\begin{Highlighting}[]
\DocumentationTok{\#\#\#\#\#\#\#\#\#\#\#\#\#\#\#\#\#\#\#\#\#\#\#\#\#\#\#\#\#\#\#\#\#\#\#\#\#\#\#\#\#\#\#\#\#\#\#\#\#\#\#\#\#\#\#\#\#\#\#\#\#\#\#\#\#\#\#\#\#\#\#\#\#\#\#\#\#\#\#\#\#\#\#\#\#\#\#\#\#\#\#\#\#\#}
\DocumentationTok{\#\# simulate}
\DocumentationTok{\#\#\#\#\#\#\#\#\#\#\#\#\#\#\#\#\#\#\#\#\#\#\#\#\#\#\#\#\#\#\#\#\#\#\#\#\#\#\#\#\#\#\#\#\#\#\#\#\#\#\#\#\#\#\#\#\#\#\#\#\#\#\#\#\#\#\#\#\#\#\#\#\#\#\#\#\#\#\#\#\#\#\#\#\#\#\#\#\#\#\#\#\#\#}
\ControlFlowTok{for}\NormalTok{ (index }\ControlFlowTok{in} \DecValTok{1}\SpecialCharTok{:}\NormalTok{(n}\SpecialCharTok{/}\DecValTok{2}\NormalTok{))}
\NormalTok{\{}
\NormalTok{  simulated\_data }\OtherTok{=} \FunctionTok{simulate}\NormalTok{(vlmcA, n.time\_series)}
\NormalTok{  vlmcs[[index]] }\OtherTok{=} \FunctionTok{list}\NormalTok{(}\DecValTok{0}\NormalTok{,}\FunctionTok{vlmc}\NormalTok{(simulated\_data, }\AttributeTok{threshold.gen=}\DecValTok{10}\NormalTok{, }\AttributeTok{alpha=}\FloatTok{0.05}\NormalTok{), }\FunctionTok{as.numeric}\NormalTok{(simulated\_data))}
\NormalTok{\}}
\ControlFlowTok{for}\NormalTok{ (index }\ControlFlowTok{in}\NormalTok{ (n}\SpecialCharTok{/}\DecValTok{2}\SpecialCharTok{+}\DecValTok{1}\NormalTok{)}\SpecialCharTok{:}\NormalTok{n)}
\NormalTok{\{}
\NormalTok{  simulated\_data }\OtherTok{=} \FunctionTok{simulate}\NormalTok{(vlmcB, n.time\_series)}
\NormalTok{  vlmcs[[index]] }\OtherTok{=} \FunctionTok{list}\NormalTok{(}\DecValTok{1}\NormalTok{,}\FunctionTok{vlmc}\NormalTok{(simulated\_data, }\AttributeTok{threshold.gen=}\DecValTok{10}\NormalTok{, }\AttributeTok{alpha=}\FloatTok{0.05}\NormalTok{), }\FunctionTok{as.numeric}\NormalTok{(simulated\_data))}
\NormalTok{\}   }

\CommentTok{\# shuffle the observations }
\NormalTok{vlmcs }\OtherTok{=}\NormalTok{ vlmcs[}\FunctionTok{sample}\NormalTok{(n)]}

\CommentTok{\# true labels of shuffled VLMCs}
\NormalTok{all\_labels }\OtherTok{=} \FunctionTok{do.call}\NormalTok{(}\StringTok{"rbind"}\NormalTok{, }\FunctionTok{lapply}\NormalTok{(vlmcs, }\StringTok{"[["}\NormalTok{, }\DecValTok{1}\NormalTok{))}
\end{Highlighting}
\end{Shaded}

After simulating the state sequences, we can compute both the structural
tDistance and the probability based pDistance to generate distance
matrices that will be used by the K-medoids algorithm. The computations
for \(D\) and \(\Delta\) are performed here.

\begin{Shaded}
\begin{Highlighting}[]
\DocumentationTok{\#\#\#\#\#\#\#\#\#\#\#\#\#\#\#\#\#\#\#\#\#\#\#\#\#\#\#\#\#\#\#\#\#\#\#\#\#\#\#\#\#\#\#\#\#\#\#\#\#\#\#\#\#\#\#\#\#\#\#\#\#\#\#\#\#\#\#\#\#\#\#\#\#\#\#\#\#\#\#\#\#\#\#\#\#\#\#\#\#\#\#\#\#\#}
\DocumentationTok{\#\# tDistance is the distance based on the contexts only}
\NormalTok{tdistanceMatrix }\OtherTok{=} \FunctionTok{matrix}\NormalTok{(}\DecValTok{0}\NormalTok{, n, n)}
\ControlFlowTok{for}\NormalTok{(index }\ControlFlowTok{in} \DecValTok{2}\SpecialCharTok{:}\NormalTok{n) }\CommentTok{\#first entry a\_11 = 0}
\NormalTok{\{}
  \ControlFlowTok{for}\NormalTok{(j }\ControlFlowTok{in} \DecValTok{1}\SpecialCharTok{:}\NormalTok{index)}
\NormalTok{    tdistanceMatrix[index,j]}\OtherTok{=} \FunctionTok{tDistance}\NormalTok{(vlmcs[[index]][[}\DecValTok{2}\NormalTok{]], vlmcs[[j]][[}\DecValTok{2}\NormalTok{]])}
\NormalTok{\}}
\NormalTok{tfulldistanceMatrix }\OtherTok{=}\NormalTok{ tdistanceMatrix }\SpecialCharTok{+} \FunctionTok{t}\NormalTok{(tdistanceMatrix)}

\DocumentationTok{\#\#\#\#\#\#\#\#\#\#\#\#\#\#\#\#\#\#\#\#\#\#\#\#\#\#\#\#\#\#\#\#\#\#\#\#\#\#\#\#\#\#\#\#\#\#\#\#\#\#\#\#\#\#\#\#\#\#\#\#\#\#\#\#\#\#\#\#\#\#\#\#\#\#\#\#\#\#\#\#\#\#\#\#\#\#\#\#\#\#\#\#\#\#}
\DocumentationTok{\#\# pDistance is the distance based on the probabilities of the reduced subgraph}
\NormalTok{pdistanceMatrix }\OtherTok{=} \FunctionTok{matrix}\NormalTok{(}\DecValTok{0}\NormalTok{, n, n)}
\ControlFlowTok{for}\NormalTok{(index }\ControlFlowTok{in} \DecValTok{2}\SpecialCharTok{:}\NormalTok{n) }\CommentTok{\#first entry a\_11 = 0}
\NormalTok{\{}
  \ControlFlowTok{for}\NormalTok{(j }\ControlFlowTok{in} \DecValTok{1}\SpecialCharTok{:}\NormalTok{index)}
\NormalTok{    pdistanceMatrix[index,j]}\OtherTok{=} \FunctionTok{pDistance}\NormalTok{(vlmcs[[index]][[}\DecValTok{2}\NormalTok{]], vlmcs[[j]][[}\DecValTok{2}\NormalTok{]], vlmcs[[index]][[}\DecValTok{3}\NormalTok{]], vlmcs[[j]][[}\DecValTok{3}\NormalTok{]])}
  
\NormalTok{\}}
\NormalTok{pdistanceMatrix[}\FunctionTok{is.na}\NormalTok{(pdistanceMatrix)] }\OtherTok{=} \DecValTok{0}
\NormalTok{pfulldistanceMatrix }\OtherTok{=}\NormalTok{ pdistanceMatrix }\SpecialCharTok{+} \FunctionTok{t}\NormalTok{(pdistanceMatrix)}
\end{Highlighting}
\end{Shaded}

\subsubsection{Clustering Specific
Distances}\label{clustering-specific-distances}

Here, we compute the distance \(D_{\alpha_{|\chi|}}^*\). Note that the
for loop over alpha divisor allows the user to investigate a set of
alpha divisor values. However, only the alpha divisor \(|\chi|^2/4\) is
used.

\begin{Shaded}
\begin{Highlighting}[]
\DocumentationTok{\#\#\#\#\#\#\#\#\#\#\#\#\#\#\#\#\#\#\#\#\#\#\#\#\#\#\#\#\#\#\#\#\#\#\#\#\#\#\#\#\#\#\#\#\#\#\#\#\#\#\#\#\#\#\#\#\#\#\#\#\#\#\#\#\#\#\#\#\#\#\#\#\#\#\#\#\#\#\#\#\#\#\#\#\#\#\#\#\#\#\#\#\#\#}
\DocumentationTok{\#\#\#\#\#\#\#\#\#\#\#\#\#\#\#\#\#\#\#\#\#\#\#\#\#\#\#\#\#\#\#\#\#\#\#\#\#\#\#\#\#\#\#\#\#\#\#\#\#\#\#\#\#\#\#\#\#\#\#\#\#\#\#\#\#\#\#\#\#\#\#\#\#\#\#\#\#\#\#\#\#\#\#\#\#\#\#\#\#\#\#\#\#\#}
\DocumentationTok{\#\# alpha homogeneity individually {-} choose alpha based on D(A,B)/alpha\_divisor {-} CLUSTERING}
\DocumentationTok{\#\#\#\#\#\#\#\#\#\#\#\#\#\#\#\#\#\#\#\#\#\#\#\#\#\#\#\#\#\#\#\#\#\#\#\#\#\#\#\#\#\#\#\#\#\#\#\#\#\#\#\#\#\#\#\#\#\#\#\#\#\#\#\#\#\#\#\#\#\#\#\#\#\#\#\#\#\#\#\#\#\#\#\#\#\#\#\#\#\#\#\#\#\#}
\DocumentationTok{\#\#\#\#\#\#\#\#\#\#\#\#\#\#\#\#\#\#\#\#\#\#\#\#\#\#\#\#\#\#\#\#\#\#\#\#\#\#\#\#\#\#\#\#\#\#\#\#\#\#\#\#\#\#\#\#\#\#\#\#\#\#\#\#\#\#\#\#\#\#\#\#\#\#\#\#\#\#\#\#\#\#\#\#\#\#\#\#\#\#\#\#\#\#}
\NormalTok{alpha\_divisor }\OtherTok{=} \FunctionTok{length}\NormalTok{(}\FunctionTok{unique}\NormalTok{(dataA))}\SpecialCharTok{\^{}}\DecValTok{2}\SpecialCharTok{/}\DecValTok{4} 
\NormalTok{homogeneity }\OtherTok{=} \DecValTok{0}

\CommentTok{\# loop is provided in case one would like to test a sequence of alpha\_divisor values }
\ControlFlowTok{for}\NormalTok{ (ind.alpha\_divisor }\ControlFlowTok{in} \DecValTok{1}\SpecialCharTok{:}\FunctionTok{length}\NormalTok{(alpha\_divisor)) }
\NormalTok{\{}
\NormalTok{  alpha }\OtherTok{=} \FunctionTok{pmin}\NormalTok{(tfulldistanceMatrix}\SpecialCharTok{/}\NormalTok{alpha\_divisor[ind.alpha\_divisor],}\DecValTok{1}\NormalTok{)}
\NormalTok{  fulldistanceMatrix }\OtherTok{=}\NormalTok{ alpha}\SpecialCharTok{*}\NormalTok{tfulldistanceMatrix }\SpecialCharTok{+}\NormalTok{ (}\DecValTok{1}\SpecialCharTok{{-}}\NormalTok{alpha)}\SpecialCharTok{*}\NormalTok{pfulldistanceMatrix}
\NormalTok{  distanceMatrix }\OtherTok{=}\NormalTok{ alpha}\SpecialCharTok{*}\NormalTok{tdistanceMatrix }\SpecialCharTok{+}\NormalTok{ (}\DecValTok{1}\SpecialCharTok{{-}}\NormalTok{alpha)}\SpecialCharTok{*}\NormalTok{pdistanceMatrix}
\NormalTok{  k\_medoids }\OtherTok{=} \FunctionTok{pam}\NormalTok{(}\FunctionTok{as.dist}\NormalTok{(distanceMatrix),}\AttributeTok{k =}\NormalTok{ K, }\AttributeTok{medoids =} \FunctionTok{sample}\NormalTok{(}\DecValTok{1}\SpecialCharTok{:}\NormalTok{n,K))}
  \DocumentationTok{\#\#\#\# compute homogeneity of clusters}
\NormalTok{  homogeneity[ind.alpha\_divisor] }\OtherTok{=} \DecValTok{0}
  \ControlFlowTok{for}\NormalTok{ (ind.k }\ControlFlowTok{in} \DecValTok{1}\SpecialCharTok{:}\NormalTok{K)}
\NormalTok{    homogeneity[ind.alpha\_divisor] }\OtherTok{=}\NormalTok{ homogeneity[ind.alpha\_divisor] }\SpecialCharTok{+} \FunctionTok{sum}\NormalTok{(fulldistanceMatrix[k\_medoids}\SpecialCharTok{$}\NormalTok{medoids[ind.k],}\FunctionTok{which}\NormalTok{(k\_medoids}\SpecialCharTok{$}\NormalTok{clustering }\SpecialCharTok{==}\NormalTok{ ind.k)])}
\NormalTok{\}}

\NormalTok{alpha\_divisor }\OtherTok{=}\NormalTok{ alpha\_divisor[}\FunctionTok{which.min}\NormalTok{(homogeneity)]}

\NormalTok{alpha }\OtherTok{=} \FunctionTok{pmin}\NormalTok{(tfulldistanceMatrix}\SpecialCharTok{/}\NormalTok{alpha\_divisor,}\DecValTok{1}\NormalTok{)}

\NormalTok{fulldistanceMatrix\_indv\_alpha }\OtherTok{=}\NormalTok{ alpha}\SpecialCharTok{*}\NormalTok{tfulldistanceMatrix }\SpecialCharTok{+}\NormalTok{ (}\DecValTok{1}\SpecialCharTok{{-}}\NormalTok{alpha)}\SpecialCharTok{*}\NormalTok{pfulldistanceMatrix}
\NormalTok{distanceMatrix\_indv\_alpha }\OtherTok{=}\NormalTok{ alpha}\SpecialCharTok{*}\NormalTok{tdistanceMatrix }\SpecialCharTok{+}\NormalTok{ (}\DecValTok{1}\SpecialCharTok{{-}}\NormalTok{alpha)}\SpecialCharTok{*}\NormalTok{pdistanceMatrix}
\end{Highlighting}
\end{Shaded}

Here, the distance \(D_{\alpha_{WCSS}}^*\) is computed.

\begin{Shaded}
\begin{Highlighting}[]
\NormalTok{alphas }\OtherTok{=} \FunctionTok{seq}\NormalTok{(}\FloatTok{0.01}\NormalTok{, }\FloatTok{0.99}\NormalTok{, }\AttributeTok{length =} \DecValTok{20}\NormalTok{)}
\NormalTok{homogeneity }\OtherTok{=} \DecValTok{0}
\ControlFlowTok{for}\NormalTok{ (ind.alphas }\ControlFlowTok{in} \DecValTok{1}\SpecialCharTok{:}\FunctionTok{length}\NormalTok{(alphas))}
\NormalTok{\{}
\NormalTok{  alpha }\OtherTok{=}\NormalTok{ alphas[ind.alphas]}
\NormalTok{  fulldistanceMatrix }\OtherTok{=}\NormalTok{ alpha}\SpecialCharTok{*}\NormalTok{tfulldistanceMatrix }\SpecialCharTok{+}\NormalTok{ (}\DecValTok{1}\SpecialCharTok{{-}}\NormalTok{alpha)}\SpecialCharTok{*}\NormalTok{pfulldistanceMatrix}
\NormalTok{  distanceMatrix }\OtherTok{=}\NormalTok{ alpha}\SpecialCharTok{*}\NormalTok{tdistanceMatrix }\SpecialCharTok{+}\NormalTok{ (}\DecValTok{1}\SpecialCharTok{{-}}\NormalTok{alpha)}\SpecialCharTok{*}\NormalTok{pdistanceMatrix}
\NormalTok{  k\_medoids }\OtherTok{=} \FunctionTok{pam}\NormalTok{(}\FunctionTok{as.dist}\NormalTok{(distanceMatrix),}\AttributeTok{k =}\NormalTok{ K, }\AttributeTok{medoids =} \FunctionTok{sample}\NormalTok{(}\DecValTok{1}\SpecialCharTok{:}\NormalTok{n,K))}
  \DocumentationTok{\#\#\#\# compute homogeneity of clusters}
\NormalTok{  homogeneity[ind.alphas] }\OtherTok{=} \DecValTok{0}
  \ControlFlowTok{for}\NormalTok{ (ind.k }\ControlFlowTok{in} \DecValTok{1}\SpecialCharTok{:}\NormalTok{K)}
\NormalTok{    homogeneity[ind.alphas] }\OtherTok{=}\NormalTok{ homogeneity[ind.alphas] }\SpecialCharTok{+} \FunctionTok{sum}\NormalTok{(fulldistanceMatrix[k\_medoids}\SpecialCharTok{$}\NormalTok{medoids[ind.k],}\FunctionTok{which}\NormalTok{(k\_medoids}\SpecialCharTok{$}\NormalTok{clustering }\SpecialCharTok{==}\NormalTok{ ind.k)])}
\NormalTok{\}}

\NormalTok{alpha }\OtherTok{=}\NormalTok{ alphas[}\FunctionTok{which.min}\NormalTok{(homogeneity)]}

\NormalTok{fulldistanceMatrix\_alpha\_fixed }\OtherTok{=}\NormalTok{ alpha}\SpecialCharTok{*}\NormalTok{tfulldistanceMatrix }\SpecialCharTok{+}\NormalTok{ (}\DecValTok{1}\SpecialCharTok{{-}}\NormalTok{alpha)}\SpecialCharTok{*}\NormalTok{pfulldistanceMatrix}
\NormalTok{distanceMatrix\_alpha\_fixed }\OtherTok{=}\NormalTok{ alpha}\SpecialCharTok{*}\NormalTok{tdistanceMatrix }\SpecialCharTok{+}\NormalTok{ (}\DecValTok{1}\SpecialCharTok{{-}}\NormalTok{alpha)}\SpecialCharTok{*}\NormalTok{pdistanceMatrix}
\end{Highlighting}
\end{Shaded}

\subsubsection{Classification Specific
Distances}\label{classification-specific-distances}

Computation of \(D_{\alpha_{|\chi|}}^*\) for classification. Note that
one could pre-specify a sequence of alpha divisor values, where the
alpha divisor that minimizes the sum of the distances to the medoids
would be selected. However, only the alpha divisor \(|\chi|^2/4\) is
used.

\begin{Shaded}
\begin{Highlighting}[]
\NormalTok{alpha\_divisor }\OtherTok{=} \FunctionTok{length}\NormalTok{(}\FunctionTok{unique}\NormalTok{(dataA))}\SpecialCharTok{\^{}}\DecValTok{2}\SpecialCharTok{/}\DecValTok{4} 
\NormalTok{homogeneity }\OtherTok{=} \DecValTok{0}
\ControlFlowTok{for}\NormalTok{ (ind.alpha\_divisor }\ControlFlowTok{in} \DecValTok{1}\SpecialCharTok{:}\FunctionTok{length}\NormalTok{(alpha\_divisor))}
\NormalTok{\{}
\NormalTok{  alpha }\OtherTok{=} \FunctionTok{pmin}\NormalTok{(tfulldistanceMatrix}\SpecialCharTok{/}\NormalTok{alpha\_divisor[ind.alpha\_divisor],}\DecValTok{1}\NormalTok{)}
\NormalTok{  fulldistanceMatrix }\OtherTok{=}\NormalTok{ alpha}\SpecialCharTok{*}\NormalTok{tfulldistanceMatrix }\SpecialCharTok{+}\NormalTok{ (}\DecValTok{1}\SpecialCharTok{{-}}\NormalTok{alpha)}\SpecialCharTok{*}\NormalTok{pfulldistanceMatrix}
\NormalTok{  distanceMatrix }\OtherTok{=}\NormalTok{ alpha}\SpecialCharTok{*}\NormalTok{tdistanceMatrix }\SpecialCharTok{+}\NormalTok{ (}\DecValTok{1}\SpecialCharTok{{-}}\NormalTok{alpha)}\SpecialCharTok{*}\NormalTok{pdistanceMatrix}
  \DocumentationTok{\#\#\#\# compute homogeneity of clusters}
\NormalTok{  homogeneity[ind.alpha\_divisor] }\OtherTok{=} \DecValTok{0}
  \ControlFlowTok{for}\NormalTok{ (ind.k }\ControlFlowTok{in} \DecValTok{1}\SpecialCharTok{:}\NormalTok{K)}
\NormalTok{  \{}
    \DocumentationTok{\#\# find the medoids of the groups from the classification labels}
\NormalTok{    med }\OtherTok{=} \FunctionTok{pam}\NormalTok{(}\FunctionTok{as.dist}\NormalTok{(distanceMatrix[}\FunctionTok{which}\NormalTok{(all\_labels }\SpecialCharTok{==}\NormalTok{ (ind.k}\DecValTok{{-}1}\NormalTok{)),}\FunctionTok{which}\NormalTok{(all\_labels }\SpecialCharTok{==}\NormalTok{ (ind.k}\DecValTok{{-}1}\NormalTok{))]),}\AttributeTok{k =} \DecValTok{1}\NormalTok{)}
\NormalTok{    homogeneity[ind.alpha\_divisor] }\OtherTok{=}\NormalTok{ homogeneity[ind.alpha\_divisor] }\SpecialCharTok{+} \FunctionTok{sum}\NormalTok{(fulldistanceMatrix[med}\SpecialCharTok{$}\NormalTok{medoids,}\FunctionTok{which}\NormalTok{(all\_labels }\SpecialCharTok{==}\NormalTok{ (ind.k}\DecValTok{{-}1}\NormalTok{))])}
\NormalTok{  \}}
\NormalTok{\}}

\NormalTok{alpha\_divisor }\OtherTok{=}\NormalTok{ alpha\_divisor[}\FunctionTok{which.min}\NormalTok{(homogeneity)]}

\NormalTok{alpha }\OtherTok{=} \FunctionTok{pmin}\NormalTok{(tfulldistanceMatrix}\SpecialCharTok{/}\NormalTok{alpha\_divisor,}\DecValTok{1}\NormalTok{)}

\NormalTok{fulldistanceMatrix\_indv\_alpha\_classif }\OtherTok{=}\NormalTok{ alpha}\SpecialCharTok{*}\NormalTok{tfulldistanceMatrix }\SpecialCharTok{+}\NormalTok{ (}\DecValTok{1}\SpecialCharTok{{-}}\NormalTok{alpha)}\SpecialCharTok{*}\NormalTok{pfulldistanceMatrix}
\NormalTok{distanceMatrix\_indv\_alpha\_classif }\OtherTok{=}\NormalTok{ alpha}\SpecialCharTok{*}\NormalTok{tdistanceMatrix }\SpecialCharTok{+}\NormalTok{ (}\DecValTok{1}\SpecialCharTok{{-}}\NormalTok{alpha)}\SpecialCharTok{*}\NormalTok{pdistanceMatrix}
\end{Highlighting}
\end{Shaded}

Computation of \(D_{\alpha_{med}}^*\). Note that we will make this a
function, since it will need to be recomputed when iterating over train
test splits (since it relies on tuning alpha with respect to the known
labels)

\begin{Shaded}
\begin{Highlighting}[]
\NormalTok{generate\_D\_alpha\_med }\OtherTok{=} \ControlFlowTok{function}\NormalTok{(tfulldistanceMatrix, pfulldistanceMatrix, all\_labels)}
\NormalTok{\{}
\NormalTok{  alphas }\OtherTok{=} \FunctionTok{seq}\NormalTok{(}\FloatTok{0.01}\NormalTok{, }\FloatTok{0.99}\NormalTok{, }\AttributeTok{length =} \DecValTok{20}\NormalTok{)}
\NormalTok{  homogeneity }\OtherTok{=} \DecValTok{0}
  \ControlFlowTok{for}\NormalTok{ (ind.alphas }\ControlFlowTok{in} \DecValTok{1}\SpecialCharTok{:}\FunctionTok{length}\NormalTok{(alphas))}
\NormalTok{  \{}
\NormalTok{    alpha }\OtherTok{=}\NormalTok{ alphas[ind.alphas]}
\NormalTok{    fulldistanceMatrix }\OtherTok{=}\NormalTok{ alpha}\SpecialCharTok{*}\NormalTok{tfulldistanceMatrix }\SpecialCharTok{+}\NormalTok{ (}\DecValTok{1}\SpecialCharTok{{-}}\NormalTok{alpha)}\SpecialCharTok{*}\NormalTok{pfulldistanceMatrix}
\NormalTok{    distanceMatrix }\OtherTok{=}\NormalTok{ alpha}\SpecialCharTok{*}\NormalTok{tdistanceMatrix }\SpecialCharTok{+}\NormalTok{ (}\DecValTok{1}\SpecialCharTok{{-}}\NormalTok{alpha)}\SpecialCharTok{*}\NormalTok{pdistanceMatrix}
    \DocumentationTok{\#\#\#\# compute homogeneity of groups of classified objects}
\NormalTok{    homogeneity[ind.alphas] }\OtherTok{=} \DecValTok{0}
    \ControlFlowTok{for}\NormalTok{ (ind.k }\ControlFlowTok{in} \DecValTok{1}\SpecialCharTok{:}\NormalTok{K)}
\NormalTok{    \{}
      \DocumentationTok{\#\# find the medoids of the groups from the classification labels}
\NormalTok{      med }\OtherTok{=} \FunctionTok{pam}\NormalTok{(}\FunctionTok{as.dist}\NormalTok{(distanceMatrix[}\FunctionTok{which}\NormalTok{(all\_labels }\SpecialCharTok{==}\NormalTok{ (ind.k}\DecValTok{{-}1}\NormalTok{)),}\FunctionTok{which}\NormalTok{(all\_labels }\SpecialCharTok{==}\NormalTok{ (ind.k}\DecValTok{{-}1}\NormalTok{))]),}\AttributeTok{k =} \DecValTok{1}\NormalTok{)}
\NormalTok{      homogeneity[ind.alphas] }\OtherTok{=}\NormalTok{ homogeneity[ind.alphas] }\SpecialCharTok{+} \FunctionTok{sum}\NormalTok{(fulldistanceMatrix[med}\SpecialCharTok{$}\NormalTok{medoids,}\FunctionTok{which}\NormalTok{(all\_labels }\SpecialCharTok{==}\NormalTok{ (ind.k}\DecValTok{{-}1}\NormalTok{))])               }
\NormalTok{    \}}
\NormalTok{  \}}
\NormalTok{  alpha }\OtherTok{=}\NormalTok{ alphas[}\FunctionTok{which.min}\NormalTok{(homogeneity)]}
  
  \FunctionTok{return}\NormalTok{(alpha)}
\NormalTok{\}}

\NormalTok{alpha }\OtherTok{=} \FunctionTok{generate\_D\_alpha\_med}\NormalTok{(tfulldistanceMatrix, pfulldistanceMatrix, all\_labels)}

\NormalTok{fulldistanceMatrix\_alpha\_fixed\_classif }\OtherTok{=}\NormalTok{ alpha}\SpecialCharTok{*}\NormalTok{tfulldistanceMatrix }\SpecialCharTok{+}\NormalTok{ (}\DecValTok{1}\SpecialCharTok{{-}}\NormalTok{alpha)}\SpecialCharTok{*}\NormalTok{pfulldistanceMatrix}
\NormalTok{distanceMatrix\_alpha\_fixed\_classif }\OtherTok{=}\NormalTok{ alpha}\SpecialCharTok{*}\NormalTok{tdistanceMatrix }\SpecialCharTok{+}\NormalTok{ (}\DecValTok{1}\SpecialCharTok{{-}}\NormalTok{alpha)}\SpecialCharTok{*}\NormalTok{pdistanceMatrix}
\end{Highlighting}
\end{Shaded}

\subsection{Classification -- corresponds to Section
3.1}\label{classification-corresponds-to-section-3.1}

\subsubsection{Leave One Out Cross Validation
(LOOCV)}\label{leave-one-out-cross-validation-loocv}

\begin{Shaded}
\begin{Highlighting}[]
\NormalTok{classif\_results }\OtherTok{=} \FunctionTok{matrix}\NormalTok{(}\AttributeTok{data =} \DecValTok{0}\NormalTok{, }\AttributeTok{nrow =} \DecValTok{5}\NormalTok{, }\AttributeTok{ncol =} \DecValTok{2}\NormalTok{)}
\FunctionTok{colnames}\NormalTok{(classif\_results) }\OtherTok{=} \FunctionTok{c}\NormalTok{(}\StringTok{"LOOCV"}\NormalTok{, }\StringTok{"70{-}30 train test"}\NormalTok{)}
\FunctionTok{rownames}\NormalTok{(classif\_results) }\OtherTok{=} \FunctionTok{c}\NormalTok{(}\StringTok{"KNN {-} tDistance"}\NormalTok{,}
                              \StringTok{"KNN {-} pDistance"}\NormalTok{,}
                              \StringTok{"KNN {-} D\_alpha\_|chi|*"}\NormalTok{,}
                              \StringTok{"KNN {-} D\_alpha\_med*"}\NormalTok{,}
                              \StringTok{"KNN {-} Classic"}
\NormalTok{)}

\CommentTok{\# tDistance }
\NormalTok{classify }\OtherTok{=} \FunctionTok{KNN\_dist\_matrix}\NormalTok{(tfulldistanceMatrix, }\DecValTok{1}\SpecialCharTok{:}\NormalTok{n, }\DecValTok{1}\SpecialCharTok{:}\NormalTok{n, }\AttributeTok{all\_labels =}\NormalTok{ all\_labels, n.neighbors)}
\NormalTok{classif\_results[}\DecValTok{1}\NormalTok{,}\DecValTok{1}\NormalTok{] }\OtherTok{=} \FunctionTok{sum}\NormalTok{(classify }\SpecialCharTok{==}\NormalTok{ all\_labels)}\SpecialCharTok{/}\NormalTok{n}

\CommentTok{\# pDistance}
\NormalTok{classify }\OtherTok{=} \FunctionTok{KNN\_dist\_matrix}\NormalTok{(pfulldistanceMatrix, }\DecValTok{1}\SpecialCharTok{:}\NormalTok{n, }\DecValTok{1}\SpecialCharTok{:}\NormalTok{n, }\AttributeTok{all\_labels =}\NormalTok{ all\_labels, n.neighbors)}
\NormalTok{classif\_results[}\DecValTok{2}\NormalTok{,}\DecValTok{1}\NormalTok{] }\OtherTok{=} \FunctionTok{sum}\NormalTok{(classify }\SpecialCharTok{==}\NormalTok{ all\_labels)}\SpecialCharTok{/}\NormalTok{n}

\CommentTok{\# D\_alpha\_|chi|*}
\NormalTok{classify }\OtherTok{=} \FunctionTok{KNN\_dist\_matrix}\NormalTok{(fulldistanceMatrix\_indv\_alpha\_classif, }\DecValTok{1}\SpecialCharTok{:}\NormalTok{n, }\DecValTok{1}\SpecialCharTok{:}\NormalTok{n, }\AttributeTok{all\_labels =}\NormalTok{ all\_labels, n.neighbors)}
\NormalTok{classif\_results[}\DecValTok{3}\NormalTok{,}\DecValTok{1}\NormalTok{] }\OtherTok{=} \FunctionTok{sum}\NormalTok{(classify }\SpecialCharTok{==}\NormalTok{ all\_labels)}\SpecialCharTok{/}\NormalTok{n}

\CommentTok{\# D\_alpha\_med*}
\CommentTok{\# need to compute alpha with respect to ONLY training data}

\NormalTok{D\_alpha\_med\_alphas }\OtherTok{=} \FunctionTok{sapply}\NormalTok{(}\DecValTok{1}\SpecialCharTok{:}\NormalTok{n, }\AttributeTok{FUN =} \ControlFlowTok{function}\NormalTok{(test\_ind)}
\NormalTok{\{}
\NormalTok{  alpha }\OtherTok{=} \FunctionTok{generate\_D\_alpha\_med}\NormalTok{(tfulldistanceMatrix[}\SpecialCharTok{{-}}\NormalTok{test\_ind, }\SpecialCharTok{{-}}\NormalTok{test\_ind],pfulldistanceMatrix[}\SpecialCharTok{{-}}\NormalTok{test\_ind, }\SpecialCharTok{{-}}\NormalTok{test\_ind], all\_labels[}\SpecialCharTok{{-}}\NormalTok{test\_ind])}
\NormalTok{\})}

\NormalTok{classify }\OtherTok{=} \FunctionTok{numeric}\NormalTok{(n)}

\CommentTok{\# get LOOCV classification results, with each alpha for D\_alpha\_med computed on current training data}
\ControlFlowTok{for}\NormalTok{(i }\ControlFlowTok{in} \DecValTok{1}\SpecialCharTok{:}\NormalTok{n)}
\NormalTok{\{}
\NormalTok{  fulldistanceMatrix\_alpha\_fixed\_classif\_curr }\OtherTok{=}\NormalTok{ D\_alpha\_med\_alphas[i]}\SpecialCharTok{*}\NormalTok{tfulldistanceMatrix }\SpecialCharTok{+}\NormalTok{ (}\DecValTok{1}\SpecialCharTok{{-}}\NormalTok{D\_alpha\_med\_alphas[i])}\SpecialCharTok{*}\NormalTok{pfulldistanceMatrix}
\NormalTok{  classify[i] }\OtherTok{=} \FunctionTok{KNN\_dist\_matrix}\NormalTok{(fulldistanceMatrix\_alpha\_fixed\_classif\_curr,}
                                \AttributeTok{train\_sample\_index =}\NormalTok{ (}\DecValTok{1}\SpecialCharTok{:}\NormalTok{n)[}\SpecialCharTok{{-}}\NormalTok{i], }
                                \AttributeTok{test\_sample\_index =}\NormalTok{ i,  }
                                \AttributeTok{all\_labels =}\NormalTok{ all\_labels, }
                                \AttributeTok{n.neighbors =}\NormalTok{ n.neighbors)}
\NormalTok{\}}

\NormalTok{classif\_results[}\DecValTok{4}\NormalTok{,}\DecValTok{1}\NormalTok{] }\OtherTok{=} \FunctionTok{sum}\NormalTok{(classify }\SpecialCharTok{==}\NormalTok{ all\_labels)}\SpecialCharTok{/}\NormalTok{n}

\CommentTok{\# Classic KNN }

\NormalTok{data\_set\_all }\OtherTok{=} \FunctionTok{t}\NormalTok{(}\FunctionTok{sapply}\NormalTok{(vlmcs, }\StringTok{"[["}\NormalTok{, }\DecValTok{3}\NormalTok{))}
\NormalTok{classify }\OtherTok{=} \FunctionTok{knn}\NormalTok{(data\_set\_all, data\_set\_all, all\_labels, n.neighbors)}
\NormalTok{classif\_results[}\DecValTok{5}\NormalTok{,}\DecValTok{1}\NormalTok{] }\OtherTok{=} \FunctionTok{sum}\NormalTok{(classify }\SpecialCharTok{==}\NormalTok{ all\_labels)}\SpecialCharTok{/}\NormalTok{n}
\end{Highlighting}
\end{Shaded}

\subsubsection{100 70-30 Train Test splits}\label{train-test-splits}

\begin{Shaded}
\begin{Highlighting}[]
\NormalTok{n.random\_splits }\OtherTok{=} \DecValTok{100}       
\ControlFlowTok{for}\NormalTok{ (index.split }\ControlFlowTok{in} \DecValTok{1}\SpecialCharTok{:}\NormalTok{n.random\_splits)}
\NormalTok{\{   }
\NormalTok{  train\_sample\_index }\OtherTok{=} \FunctionTok{sample}\NormalTok{(}\DecValTok{1}\SpecialCharTok{:}\NormalTok{n,n}\SpecialCharTok{*}\FloatTok{0.70}\NormalTok{)}
\NormalTok{  test\_sample\_index }\OtherTok{=}\NormalTok{ (}\DecValTok{1}\SpecialCharTok{:}\NormalTok{n)[}\SpecialCharTok{{-}}\NormalTok{train\_sample\_index]}
  
\NormalTok{  train\_vlmcs }\OtherTok{=}\NormalTok{ vlmcs[train\_sample\_index]}
\NormalTok{  test\_vlmcs }\OtherTok{=}\NormalTok{ vlmcs[test\_sample\_index]}
\NormalTok{  true\_labels\_train }\OtherTok{=} \FunctionTok{do.call}\NormalTok{(}\StringTok{"rbind"}\NormalTok{, }\FunctionTok{lapply}\NormalTok{(train\_vlmcs, }\StringTok{"[["}\NormalTok{, }\DecValTok{1}\NormalTok{))}
\NormalTok{  true\_labels\_test }\OtherTok{=} \FunctionTok{do.call}\NormalTok{(}\StringTok{"rbind"}\NormalTok{, }\FunctionTok{lapply}\NormalTok{(test\_vlmcs, }\StringTok{"[["}\NormalTok{, }\DecValTok{1}\NormalTok{))}
  
  \CommentTok{\# tDistance }
\NormalTok{  classify }\OtherTok{=} \FunctionTok{KNN\_dist\_matrix}\NormalTok{(tfulldistanceMatrix, train\_sample\_index, test\_sample\_index, }\AttributeTok{all\_labels =}\NormalTok{ all\_labels, n.neighbors)}
\NormalTok{  classif\_results[}\DecValTok{1}\NormalTok{,}\DecValTok{2}\NormalTok{] }\OtherTok{=}\NormalTok{ classif\_results[}\DecValTok{1}\NormalTok{,}\DecValTok{2}\NormalTok{] }\SpecialCharTok{+} \FunctionTok{sum}\NormalTok{(classify }\SpecialCharTok{==}\NormalTok{ true\_labels\_test)}\SpecialCharTok{/}\FunctionTok{length}\NormalTok{(test\_sample\_index)}
  
  \CommentTok{\# pDistance}
\NormalTok{  classify }\OtherTok{=} \FunctionTok{KNN\_dist\_matrix}\NormalTok{(pfulldistanceMatrix, train\_sample\_index, test\_sample\_index, }\AttributeTok{all\_labels =}\NormalTok{ all\_labels, n.neighbors)}
\NormalTok{  classif\_results[}\DecValTok{2}\NormalTok{,}\DecValTok{2}\NormalTok{] }\OtherTok{=}\NormalTok{ classif\_results[}\DecValTok{2}\NormalTok{,}\DecValTok{2}\NormalTok{] }\SpecialCharTok{+} \FunctionTok{sum}\NormalTok{(classify }\SpecialCharTok{==}\NormalTok{ true\_labels\_test)}\SpecialCharTok{/}\FunctionTok{length}\NormalTok{(test\_sample\_index)}
  
  \CommentTok{\# D\_alpha\_|chi|*}
\NormalTok{  classify }\OtherTok{=} \FunctionTok{KNN\_dist\_matrix}\NormalTok{(fulldistanceMatrix\_indv\_alpha\_classif, train\_sample\_index, test\_sample\_index, }\AttributeTok{all\_labels =}\NormalTok{ all\_labels, n.neighbors)}
\NormalTok{  classif\_results[}\DecValTok{3}\NormalTok{,}\DecValTok{2}\NormalTok{] }\OtherTok{=}\NormalTok{ classif\_results[}\DecValTok{3}\NormalTok{,}\DecValTok{2}\NormalTok{] }\SpecialCharTok{+} \FunctionTok{sum}\NormalTok{(classify }\SpecialCharTok{==}\NormalTok{ true\_labels\_test)}\SpecialCharTok{/}\FunctionTok{length}\NormalTok{(test\_sample\_index)}
  
  \CommentTok{\# D\_alpha\_med*}
  \CommentTok{\# need to compute alpha with respect to ONLY training data}
\NormalTok{  curr\_alpha }\OtherTok{=} \FunctionTok{generate\_D\_alpha\_med}\NormalTok{(tfulldistanceMatrix[}\SpecialCharTok{{-}}\NormalTok{test\_sample\_index, }\SpecialCharTok{{-}}\NormalTok{test\_sample\_index], }
\NormalTok{                                    pfulldistanceMatrix[}\SpecialCharTok{{-}}\NormalTok{test\_sample\_index, }\SpecialCharTok{{-}}\NormalTok{test\_sample\_index], }
                                    \AttributeTok{all\_labels =}\NormalTok{ all\_labels[}\SpecialCharTok{{-}}\NormalTok{test\_sample\_index])}
  
\NormalTok{  fulldistanceMatrix\_alpha\_fixed\_classif\_curr }\OtherTok{=}\NormalTok{ curr\_alpha}\SpecialCharTok{*}\NormalTok{tfulldistanceMatrix }\SpecialCharTok{+}\NormalTok{ (}\DecValTok{1}\SpecialCharTok{{-}}\NormalTok{curr\_alpha)}\SpecialCharTok{*}\NormalTok{pfulldistanceMatrix}
  
\NormalTok{  classify }\OtherTok{=} \FunctionTok{KNN\_dist\_matrix}\NormalTok{(fulldistanceMatrix\_alpha\_fixed\_classif\_curr, train\_sample\_index, test\_sample\_index, }\AttributeTok{all\_labels =}\NormalTok{ all\_labels, n.neighbors)}
\NormalTok{  classif\_results[}\DecValTok{4}\NormalTok{,}\DecValTok{2}\NormalTok{] }\OtherTok{=}\NormalTok{ classif\_results[}\DecValTok{4}\NormalTok{,}\DecValTok{2}\NormalTok{] }\SpecialCharTok{+} \FunctionTok{sum}\NormalTok{(classify }\SpecialCharTok{==}\NormalTok{ true\_labels\_test)}\SpecialCharTok{/}\FunctionTok{length}\NormalTok{(test\_sample\_index)}
  
  \CommentTok{\# Classic KNN }
  
  \CommentTok{\# get training state sequences }
  
\NormalTok{  data\_set\_train }\OtherTok{=} \FunctionTok{t}\NormalTok{(}\FunctionTok{sapply}\NormalTok{(train\_vlmcs, }\StringTok{"[["}\NormalTok{, }\DecValTok{3}\NormalTok{))}
\NormalTok{  data\_set\_test }\OtherTok{=} \FunctionTok{t}\NormalTok{(}\FunctionTok{sapply}\NormalTok{(test\_vlmcs, }\StringTok{"[["}\NormalTok{, }\DecValTok{3}\NormalTok{))}
  
\NormalTok{  classify }\OtherTok{=} \FunctionTok{knn}\NormalTok{(data\_set\_train, data\_set\_test, true\_labels\_train, n.neighbors)}
\NormalTok{  classif\_results[}\DecValTok{5}\NormalTok{,}\DecValTok{2}\NormalTok{] }\OtherTok{=}\NormalTok{ classif\_results[}\DecValTok{5}\NormalTok{,}\DecValTok{2}\NormalTok{] }\SpecialCharTok{+} \FunctionTok{sum}\NormalTok{(classify }\SpecialCharTok{==}\NormalTok{ true\_labels\_test)}\SpecialCharTok{/}\FunctionTok{length}\NormalTok{(test\_sample\_index)}
\NormalTok{\}}


\CommentTok{\# divide by simulation replicates for 100 70{-}30 train test splits }
\NormalTok{classif\_results[,}\DecValTok{2}\NormalTok{] }\OtherTok{=}\NormalTok{ classif\_results[,}\DecValTok{2}\NormalTok{] }\SpecialCharTok{/}\NormalTok{ n.random\_splits}

\FunctionTok{cat}\NormalTok{(}\FunctionTok{paste0}\NormalTok{(}\StringTok{"Classification Results for n.time\_series = "}\NormalTok{, n.time\_series, }\StringTok{", n = "}\NormalTok{,n, }\StringTok{":}\SpecialCharTok{\textbackslash{}n}\StringTok{"}\NormalTok{))}
\end{Highlighting}
\end{Shaded}

\begin{verbatim}
## Classification Results for n.time_series = 500, n = 80:
\end{verbatim}

\begin{Shaded}
\begin{Highlighting}[]
\NormalTok{classif\_results}
\end{Highlighting}
\end{Shaded}

\begin{verbatim}
##                       LOOCV 70-30 train test
## KNN - tDistance      1.0000        0.9870833
## KNN - pDistance      1.0000        0.9991667
## KNN - D_alpha_|chi|* 1.0000        1.0000000
## KNN - D_alpha_med*   1.0000        1.0000000
## KNN - Classic        0.7375        0.6020833
\end{verbatim}

\subsection{Clustering -- corresponds to Section
3.2}\label{clustering-corresponds-to-section-3.2}

\begin{Shaded}
\begin{Highlighting}[]
\CommentTok{\# number of simulation replicates.  Note that multiple simulation replicates are }
\CommentTok{\# run to average over Kmedoid and K means cluster performance from random initialization }
\CommentTok{\# of centroids }
\NormalTok{n.iter }\OtherTok{=} \DecValTok{100}

\CommentTok{\# the following are initialized for storing results}
\NormalTok{clustering\_results }\OtherTok{=} \FunctionTok{matrix}\NormalTok{(}\DecValTok{0}\NormalTok{, }\AttributeTok{nrow =} \DecValTok{5}\NormalTok{, }\AttributeTok{ncol =} \DecValTok{3}\NormalTok{)}
\FunctionTok{colnames}\NormalTok{(clustering\_results) }\OtherTok{=} \FunctionTok{c}\NormalTok{(}\StringTok{"rate"}\NormalTok{, }\StringTok{"rand.index\_rate"}\NormalTok{,}\StringTok{"mutual\_info rate"}\NormalTok{)}
\FunctionTok{rownames}\NormalTok{(clustering\_results) }\OtherTok{=} \FunctionTok{c}\NormalTok{(}\StringTok{"K{-}Medoids {-} tDistance"}\NormalTok{, }
                                 \StringTok{"K{-}Medoids {-} pDistance"}\NormalTok{,}
                                 \StringTok{"K{-}Medoids {-} D\_\{alpha\_\{|chi|\}\}*"}\NormalTok{, }
                                 \StringTok{"K{-}Medoids {-} D\_\{alpha\_\{WCSS\}\}*"}\NormalTok{, }
                                 \StringTok{"Classical KNN"}\NormalTok{)}



\CommentTok{\# to compare with K means clustering, we need to extract the state sequences.  }
\CommentTok{\# They will be treated as points with n.time\_series dimensions in KNN. }
\NormalTok{state\_sequences }\OtherTok{=} \FunctionTok{t}\NormalTok{(}\FunctionTok{sapply}\NormalTok{(vlmcs, }\StringTok{"[["}\NormalTok{, }\DecValTok{3}\NormalTok{))}

\CommentTok{\# set seed for reproducibility }
\FunctionTok{set.seed}\NormalTok{(}\DecValTok{2024}\NormalTok{)}


\CommentTok{\# Iterate over n.iter iterations {-}{-} accounting for randomness in initial centroid selection }
\ControlFlowTok{for}\NormalTok{ (iter }\ControlFlowTok{in} \DecValTok{1}\SpecialCharTok{:}\NormalTok{n.iter) }
\NormalTok{\{}
  \CommentTok{\# K medoids using tDistance}
\NormalTok{  k\_medoids }\OtherTok{=} \FunctionTok{pam}\NormalTok{(}\AttributeTok{x =} \FunctionTok{as.dist}\NormalTok{(tdistanceMatrix),}\AttributeTok{k =}\NormalTok{ K, }\AttributeTok{medoids =} \FunctionTok{sample}\NormalTok{(}\DecValTok{1}\SpecialCharTok{:}\NormalTok{n,K))}
\NormalTok{  clustering\_results[}\DecValTok{1}\NormalTok{,}\DecValTok{1}\NormalTok{] }\OtherTok{=}\NormalTok{ clustering\_results[}\DecValTok{1}\NormalTok{,}\DecValTok{1}\NormalTok{] }\SpecialCharTok{+}\NormalTok{ (n }\SpecialCharTok{{-}} \FunctionTok{Mismatch}\NormalTok{(k\_medoids}\SpecialCharTok{$}\NormalTok{clustering, (}\FunctionTok{as.numeric}\NormalTok{(all\_labels)}\SpecialCharTok{+}\DecValTok{1}\NormalTok{), }\AttributeTok{K =}\NormalTok{ K))}\SpecialCharTok{/}\NormalTok{n}
\NormalTok{  clustering\_results[}\DecValTok{1}\NormalTok{,}\DecValTok{2}\NormalTok{] }\OtherTok{=}\NormalTok{ clustering\_results[}\DecValTok{1}\NormalTok{,}\DecValTok{2}\NormalTok{] }\SpecialCharTok{+} \FunctionTok{rand.index}\NormalTok{(k\_medoids}\SpecialCharTok{$}\NormalTok{clustering, (}\FunctionTok{as.numeric}\NormalTok{(all\_labels)}\SpecialCharTok{+}\DecValTok{1}\NormalTok{))}
\NormalTok{  clustering\_results[}\DecValTok{1}\NormalTok{,}\DecValTok{3}\NormalTok{] }\OtherTok{=}\NormalTok{ clustering\_results[}\DecValTok{1}\NormalTok{,}\DecValTok{3}\NormalTok{] }\SpecialCharTok{+} \FunctionTok{mutinformation}\NormalTok{(k\_medoids}\SpecialCharTok{$}\NormalTok{clustering, (}\FunctionTok{as.numeric}\NormalTok{(all\_labels)}\SpecialCharTok{+}\DecValTok{1}\NormalTok{))}
  
  \CommentTok{\# K medioids using pDistance }
\NormalTok{  k\_medoids }\OtherTok{=} \FunctionTok{pam}\NormalTok{(}\AttributeTok{x =} \FunctionTok{as.dist}\NormalTok{(pdistanceMatrix),}\AttributeTok{k =}\NormalTok{ K, }\AttributeTok{medoids =} \FunctionTok{sample}\NormalTok{(}\DecValTok{1}\SpecialCharTok{:}\NormalTok{n,K))}
\NormalTok{  clustering\_results[}\DecValTok{2}\NormalTok{,}\DecValTok{1}\NormalTok{] }\OtherTok{=}\NormalTok{ clustering\_results[}\DecValTok{2}\NormalTok{,}\DecValTok{1}\NormalTok{] }\SpecialCharTok{+}\NormalTok{ (n }\SpecialCharTok{{-}} \FunctionTok{Mismatch}\NormalTok{(k\_medoids}\SpecialCharTok{$}\NormalTok{clustering, (}\FunctionTok{as.numeric}\NormalTok{(all\_labels)}\SpecialCharTok{+}\DecValTok{1}\NormalTok{), }\AttributeTok{K =}\NormalTok{ K))}\SpecialCharTok{/}\NormalTok{n}
\NormalTok{  clustering\_results[}\DecValTok{2}\NormalTok{,}\DecValTok{2}\NormalTok{] }\OtherTok{=}\NormalTok{ clustering\_results[}\DecValTok{2}\NormalTok{,}\DecValTok{2}\NormalTok{] }\SpecialCharTok{+} \FunctionTok{rand.index}\NormalTok{(k\_medoids}\SpecialCharTok{$}\NormalTok{clustering, (}\FunctionTok{as.numeric}\NormalTok{(all\_labels)}\SpecialCharTok{+}\DecValTok{1}\NormalTok{))}
\NormalTok{  clustering\_results[}\DecValTok{2}\NormalTok{,}\DecValTok{3}\NormalTok{] }\OtherTok{=}\NormalTok{ clustering\_results[}\DecValTok{2}\NormalTok{,}\DecValTok{3}\NormalTok{] }\SpecialCharTok{+} \FunctionTok{mutinformation}\NormalTok{(k\_medoids}\SpecialCharTok{$}\NormalTok{clustering, (}\FunctionTok{as.numeric}\NormalTok{(all\_labels)}\SpecialCharTok{+}\DecValTok{1}\NormalTok{))}
  
  \CommentTok{\# K medoids using D\_\{\textbackslash{}alpha\_\{|\textbackslash{}chi|\}\}\^{}*}
\NormalTok{  k\_medoids }\OtherTok{=} \FunctionTok{pam}\NormalTok{(}\FunctionTok{as.dist}\NormalTok{(distanceMatrix\_indv\_alpha),}\AttributeTok{k =}\NormalTok{ K, }\AttributeTok{medoids =} \FunctionTok{sample}\NormalTok{(}\DecValTok{1}\SpecialCharTok{:}\NormalTok{n,K))}
\NormalTok{  clustering\_results[}\DecValTok{3}\NormalTok{,}\DecValTok{1}\NormalTok{] }\OtherTok{=}\NormalTok{ clustering\_results[}\DecValTok{3}\NormalTok{,}\DecValTok{1}\NormalTok{] }\SpecialCharTok{+}\NormalTok{ (n }\SpecialCharTok{{-}} \FunctionTok{Mismatch}\NormalTok{(k\_medoids}\SpecialCharTok{$}\NormalTok{clustering, (}\FunctionTok{as.numeric}\NormalTok{(all\_labels)}\SpecialCharTok{+}\DecValTok{1}\NormalTok{), }\AttributeTok{K =}\NormalTok{ K))}\SpecialCharTok{/}\NormalTok{n}
\NormalTok{  clustering\_results[}\DecValTok{3}\NormalTok{,}\DecValTok{2}\NormalTok{] }\OtherTok{=}\NormalTok{ clustering\_results[}\DecValTok{3}\NormalTok{,}\DecValTok{2}\NormalTok{] }\SpecialCharTok{+} \FunctionTok{rand.index}\NormalTok{(k\_medoids}\SpecialCharTok{$}\NormalTok{clustering, (}\FunctionTok{as.numeric}\NormalTok{(all\_labels)}\SpecialCharTok{+}\DecValTok{1}\NormalTok{))}
\NormalTok{  clustering\_results[}\DecValTok{3}\NormalTok{,}\DecValTok{3}\NormalTok{] }\OtherTok{=}\NormalTok{ clustering\_results[}\DecValTok{3}\NormalTok{,}\DecValTok{3}\NormalTok{] }\SpecialCharTok{+} \FunctionTok{mutinformation}\NormalTok{(k\_medoids}\SpecialCharTok{$}\NormalTok{clustering, (}\FunctionTok{as.numeric}\NormalTok{(all\_labels)}\SpecialCharTok{+}\DecValTok{1}\NormalTok{))}
  
  \CommentTok{\# K medoids using D\_\{\textbackslash{}alpha\_\{WCSS\}\}\^{}*}
\NormalTok{  k\_medoids }\OtherTok{=} \FunctionTok{pam}\NormalTok{(}\FunctionTok{as.dist}\NormalTok{(distanceMatrix\_alpha\_fixed),}\AttributeTok{k =}\NormalTok{ K, }\AttributeTok{medoids =} \FunctionTok{sample}\NormalTok{(}\DecValTok{1}\SpecialCharTok{:}\NormalTok{n,K))}
\NormalTok{  clustering\_results[}\DecValTok{4}\NormalTok{,}\DecValTok{1}\NormalTok{] }\OtherTok{=}\NormalTok{ clustering\_results[}\DecValTok{4}\NormalTok{,}\DecValTok{1}\NormalTok{] }\SpecialCharTok{+}\NormalTok{ (n }\SpecialCharTok{{-}} \FunctionTok{Mismatch}\NormalTok{(k\_medoids}\SpecialCharTok{$}\NormalTok{clustering, (}\FunctionTok{as.numeric}\NormalTok{(all\_labels)}\SpecialCharTok{+}\DecValTok{1}\NormalTok{), }\AttributeTok{K =}\NormalTok{ K))}\SpecialCharTok{/}\NormalTok{n}
\NormalTok{  clustering\_results[}\DecValTok{4}\NormalTok{,}\DecValTok{2}\NormalTok{] }\OtherTok{=}\NormalTok{ clustering\_results[}\DecValTok{4}\NormalTok{,}\DecValTok{2}\NormalTok{] }\SpecialCharTok{+} \FunctionTok{rand.index}\NormalTok{(k\_medoids}\SpecialCharTok{$}\NormalTok{clustering, (}\FunctionTok{as.numeric}\NormalTok{(all\_labels)}\SpecialCharTok{+}\DecValTok{1}\NormalTok{))}
\NormalTok{  clustering\_results[}\DecValTok{4}\NormalTok{,}\DecValTok{3}\NormalTok{] }\OtherTok{=}\NormalTok{ clustering\_results[}\DecValTok{4}\NormalTok{,}\DecValTok{3}\NormalTok{] }\SpecialCharTok{+} \FunctionTok{mutinformation}\NormalTok{(k\_medoids}\SpecialCharTok{$}\NormalTok{clustering, (}\FunctionTok{as.numeric}\NormalTok{(all\_labels)}\SpecialCharTok{+}\DecValTok{1}\NormalTok{))}
  
  \CommentTok{\# Classical K means }
  
\NormalTok{  classic\_kmeans }\OtherTok{=} \FunctionTok{kmeans}\NormalTok{(state\_sequences, }\AttributeTok{centers =}\NormalTok{ K)}
\NormalTok{  clustering\_results[}\DecValTok{5}\NormalTok{,}\DecValTok{1}\NormalTok{] }\OtherTok{=}\NormalTok{ clustering\_results[}\DecValTok{5}\NormalTok{,}\DecValTok{1}\NormalTok{] }\SpecialCharTok{+}\NormalTok{ (n }\SpecialCharTok{{-}} \FunctionTok{Mismatch}\NormalTok{(classic\_kmeans}\SpecialCharTok{$}\NormalTok{cluster, (}\FunctionTok{as.numeric}\NormalTok{(all\_labels)}\SpecialCharTok{+}\DecValTok{1}\NormalTok{), }\AttributeTok{K =}\NormalTok{ K))}\SpecialCharTok{/}\NormalTok{n}
\NormalTok{  clustering\_results[}\DecValTok{5}\NormalTok{,}\DecValTok{2}\NormalTok{] }\OtherTok{=}\NormalTok{ clustering\_results[}\DecValTok{5}\NormalTok{,}\DecValTok{2}\NormalTok{] }\SpecialCharTok{+} \FunctionTok{rand.index}\NormalTok{(classic\_kmeans}\SpecialCharTok{$}\NormalTok{cluster, (}\FunctionTok{as.numeric}\NormalTok{(all\_labels)}\SpecialCharTok{+}\DecValTok{1}\NormalTok{))}
\NormalTok{  clustering\_results[}\DecValTok{5}\NormalTok{,}\DecValTok{3}\NormalTok{] }\OtherTok{=}\NormalTok{ clustering\_results[}\DecValTok{5}\NormalTok{,}\DecValTok{3}\NormalTok{] }\SpecialCharTok{+} \FunctionTok{mutinformation}\NormalTok{(classic\_kmeans}\SpecialCharTok{$}\NormalTok{cluster, (}\FunctionTok{as.numeric}\NormalTok{(all\_labels)}\SpecialCharTok{+}\DecValTok{1}\NormalTok{))}
\NormalTok{\}}

\CommentTok{\# divide by simulation replicates }
\NormalTok{clustering\_results }\OtherTok{=}\NormalTok{ clustering\_results }\SpecialCharTok{/}\NormalTok{ n.iter}

\FunctionTok{cat}\NormalTok{(}\FunctionTok{paste0}\NormalTok{(}\StringTok{"Clustering Results for n.time\_series = "}\NormalTok{, n.time\_series, }\StringTok{", n = "}\NormalTok{,n, }\StringTok{":}\SpecialCharTok{\textbackslash{}n}\StringTok{"}\NormalTok{))}
\end{Highlighting}
\end{Shaded}

\begin{verbatim}
## Clustering Results for n.time_series = 500, n = 80:
\end{verbatim}

\begin{Shaded}
\begin{Highlighting}[]
\NormalTok{clustering\_results }
\end{Highlighting}
\end{Shaded}

\begin{verbatim}
##                                    rate rand.index_rate mutual_info rate
## K-Medoids - tDistance          0.899500       0.8368734      0.469200600
## K-Medoids - pDistance          0.975000       0.9506329      0.592639038
## K-Medoids - D_{alpha_{|chi|}}* 1.000000       1.0000000      0.693147181
## K-Medoids - D_{alpha_{WCSS}}*  0.975000       0.9506329      0.592639038
## Classical KNN                  0.543625       0.4989209      0.005268734
\end{verbatim}

\end{document}
